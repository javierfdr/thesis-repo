\documentclass{article}
\usepackage{geometry} % see geometry.pdf on how to lay out the page. There's lots.
\geometry{letterpaper} % or letter or a5paper or ... etc
% \geometry{landscape} % rotated page geometry
\RequirePackage{amsmath}
\usepackage{listings}

\usepackage{graphicx}
\usepackage{amssymb}
\usepackage{epstopdf}
\usepackage{longtable}
\usepackage{hyperref}

\begin{document}

\title{Master Thesis Project Specification\\
\textbf{Image processing to detect worms}}
\author{Javier Fern\'andez}
%\date{17 de Febrero de 2009} % delete this line to display the current date
\maketitle



\section{Background}
Images of biological samples are no longer just overview pictures; 
they are measurements. To turn images into manageable data the computer 
has to be able to make sense of them. This involves processing the raw 
images to gather and organize data that would allow to analyze and 
manipulate the information accurately and faster.\\

 The purpose of this project is to detect C.elegans worms (larva) in 
liquid media. It would be implemented as part of the 
image analysis and data processing software \emph{Endrov}. 
The project can be extended to allow tracking 
(using microscope XY stage) of worms moving on plates if time allows.

\section{Task description}
The general objective is to obtain an algorithm that receives images of
worms in a liquid culture as input and outputs fitted shapes of worms.
This will involve two main stages: the implementation stage and the fine-tune
and benchmarking stage.\\

\subsection{Implementation stage}
In this stage will be studied the previous attempts and approaches
on the different algorithms that have to be implemented in order to fulfil
the project main objective of fitting the worms shape from an initial
microscope image.\\

The implementation would be done as a filter for \emph{Endrov}, an 
image analysis and data processing software developed at Karolinska
Institute. This open source project provides a wide suite of tools that
allows to manipulate biological data with agility, making more efficient
the study of samples.\\

The main activities that the project involve are the following:
\begin{itemize}
\item Rasterizing and maybe tesselating general polygons.
\item Finding a suitable thresholding algorithm. This will allow to obtain a 
binary image from the initial sample, dividing the images into
object pixels and background pixels. This will facilitate the manipulation 
and filtering of the data.
\item Finding a good shape descriptor. This involves researching on different
past attempts on shape fitting, and designing a shape descriptor that fits
the current problem. There are many published studies on shape fitting 
for many different problems, but just few focused on worms.
\item Doing the math required to use a normal continuous optimization algorithm
which does not require differentials.
\item Optimizing code, both data structures and constant time factor.
\end{itemize}

\subsection{Benchmarking and fine tuning}
This stage goal is to analyze and test the accuracy of the implemented
approach. The main activities are:

\begin{itemize}
\item Benchmarking algorithm with expert annotated images.
\item Fine tuning the code to suit better the specification and improve
  unnacurate results.
\end{itemize}

\section{Procedure}

\subsection{Technical fields addresed}
\begin{itemize}
\item Numerical Optimization (use off-the shelf algorithm)
\item Image processing (thresholding, distance transsformation, skeletonization)
\item Computer Graphics (rendering)
\item Code Optimization
\item Algorithms, data structures
\item Interpolation (maybe splines)
\end{itemize}

\subsection{Programming language and environment}
\begin{itemize}
\item Java on Eclipse. (Linux environment)
\item Working with a large source code with GIT version control (\emph{Endrov}) 
\end{itemize}

\subsection{Documentation}
\begin{itemize}
\item Weekly report highlighting problems addressed, new solutions and 
  problems found, references contribution and links.
\end{itemize}

\section{Time Schedule}
\begin{itemize}
\item Finding a good thresholding algorithm, 2w
\item Rasterizer and polygon ROI, 2w
\item Implement shape descriptor, 4w
\item Implement optimizer (use a library if possible), 3w
\item Misc helper image processing fuctions, 1w
\item Fine-tune and benchmark algorithm. Some way of guessing initial shape, 5w
\item Report writing and occasional meetings, 3w. (The project is being
  written in parallel with every task. There are 3 weeks more for completing
  the report and refining).
\end{itemize}

\subsection{Delimitations}
\begin{itemize}
\item The implementations tasks are straight forward. The minimum objective
is the full implementation.
\item The benchmarking and fine tuning step can fail entirely.
\end{itemize}

 Minimum objective is an attempt and if it does not work, 
 documentation of what the problems were and suggestions for further work.


\end{document}