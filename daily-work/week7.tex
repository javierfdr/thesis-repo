\documentclass{article}
\usepackage{geometry} % see geometry.pdf on how to lay out the page. There's lots.
\geometry{letterpaper} % or letter or a5paper or ... etc
% \geometry{landscape} % rotated page geometry
\RequirePackage{amsmath}
\usepackage{listings}

\usepackage{graphicx}
\usepackage{amssymb}
\usepackage{epstopdf}
\usepackage{longtable}
\usepackage{hyperref}
\usepackage{color}
\usepackage{listings}
\usepackage{algorithmic}
\usepackage{algorithm}

\title{Daily Thesis-work Report}
\author{Javier Fern\'andez}
%\date{17 de Febrero de 2009} % delete this line to display the current date

\begin{document}

\maketitle
%\newpage
%\tableofcontents
%\listoffigures
%\listoftables
%\newpage

\section{Note}
Nice description and reference on \emph{C.Elegans} usefulness on
\emph{Strategies for automated anlysis of C.Elegans}

\section{On Energy formulation}
\subsection*{Johan considerations}
\begin{algorithm}                      
\caption{Formulation of the optimization algorithm}         
\label{alg1}                          
\begin{algorithmic}                   
\REQUIRE $wc \gets $ WormClusterSkeleton
\STATE $path \gets$ initialGuess($wc$)
\WHILE{not($convergence$)}
\STATE $path \gets$ perturbateCP($path$)
\STATE $energy \gets$ energyFormulation($angles$,$maskRender$)
\IF{$energy < prevEnergy$}
\STATE keepNewCP()
\ELSE
\STATE takePreviousCP()
\ENDIF
\ENDWHILE
\end{algorithmic}
\end{algorithm}

\subsection{week10-snakes.ppt: Snakes or active contours}
\begin{itemize}
\item Seek perimeter of region
\item Perimeter is constrained by factors
\item Smoothnes or bending energy
\item Fit to image gradient
\item Fit to fixed points in image
\item Define all factors mathematically and use
  optimization to find the best perimeter
\item Variational search needs starting point
\end{itemize}

\subsection{Snakes: Active contour models}
A snake is an energy-minimizing     spline guided by external 
constraint forces and influenced by image forces that pull it toward 
features such as lines and edges. Snakes are active contour models: they lock
 onto nearby edges, localizing them accurately. Scale-space continuation can
 be used to enlarge the capture region surrounding a feature. Snakes provide 
a unified account of a number of visual problems, including detection of 
edges, lines, and subjective contours; motion tracking; and stereo matching.\\

We seek to design energy functions whose local minima comprise the set of 
alternative solutions available to higher-level processes.
By ading suitable energy terms to the minimization, it is possible for a user
to push the model out of a local minimum toward the desired solution. The result
is an active model that falls into the desired solution when placed near it.\\

ME: energy minimizing models have tipically been regarded as autonomous, but
they have developed interactive techniques for guiding them.( Look second
paragraph second column page 1)\\

They provide an interact guide to push a contour model to fit the desired
shape: High level mechanisms can interact with the contour model by pushing it
toward an appropriate local minimum. Optimization and relaxatoin have 
been used previously in edge and line detection, but without the interactive
guiding used here.\\

Because of the way the contours slither while minimizing their energy, we 
call them snakes.\\

\textbf{The energies}: Our basic snake model is a controlled continuity
spline under the influence of image forces and external constraint forces.
The internal spline forces serve to impose a piecewise smoothness constraint.
The image forces push the snake toward salient image features like lines,edges,
and subjective contours. The external constraint forces are responsible for 
putting the snake near the desired local minimum.


\subsection{Deformable models}
good introduction about deformable models to fit shapes\\

   All algorithms for matching a deformation model to a given data set are defined as an energy minimization problem. Some
measure of how well the deformed model matches the data has to be minimized. We call this \textbf{the external energy} that pushes
the model to match the data set as good as possible. At the same time the \textbf{internal energy}, representing the prior knowledge,
has to be kept as low as possible. \textbf{The internal energy} models the object’s resistance to be pushed by the \textbf{external} force into
directions not coherent with the prior knowledge. \\

forces. For instance, in the case of Snakes, this means that a contour is pushed to an image feature by the external force while
the contour itself exhibits resistance to be deformed into a non-smooth curve. In the case of the 3DMM, the internal forces
become strong when the object is deformed such that it does not belong to the correct object class. 
\textbf{Important to notice: } After these considerations it seems that it is
not necessary to consider internal energy or something similar because the
control point generated shape will never separate from the correct object
class: is done with that purpose.\\

Let $M$ be the model and $D$ a data set the energy $E$ can be 
defined as:
$$E(M) = E_{ext}(M,D) + E_{int}(M)$$

\subsubsection*{External Energy}
The external energy measures how well the snake matches the 
boundaries in the image. Using an edge image $I$ with low
values on the edged of the images (distance transformation 
for example) the external energy is given as:

$$E_{ext}(M,D) = E_{ext}(v,I) = \int_0^1 I(v(s))ds$$

                                                                   Therefore, the external energy is minimized if the curve 
comes to lie completely on a boundary of an image.

\section{Formulation Ideas}

\subsection*{Permutation}
\begin{itemize}
\item The best permutation possible seems to be angles between segments
\item The control point would be repositioned according to the new angle.
  The new angle could be either the same segments or the mirror. This is
  because the same angle can have to different middle control point positions
  depending on the bending orientation.
\item The different positions of the middle control point seems to be those
  along the bisection line
\item Another possibility is move in surrounding circles (the radio that
  connects the control points, but its real complicated)
\end{itemize}

\subsection*{Energy formulation}


\begin{itemize}
\item The external energy would be measured masking the thresholding image
  and the generated shape.This is: the external energy are the amount of
  points in the generated shape that are not object pixels.
\item The internal energy is supposed to help keeping the shape \emph{class}
  or predefined structure. Then, it shouldn't be necessary to define it 
  in this formulation
\item Somehow the angles between segments could be an important thing to
  take into account on the energy formulation
\end{itemize}

\subsection*{Shape Deformation}
Taking the angles approach the neighborhood of a given control points skeleton
with fixed extreme points could be calculated through the variation of those
angles. Here some research ideas about \textbf{polyline}, 
\textbf{polyline bending} and others.

\begin{itemize}
\item 
\end{itemize}


\end{document}


