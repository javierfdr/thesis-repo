%\cleardoublepage
\thispagestyle{empty}
\chapter{INTRODUCCI\'ON}
\pagenumbering{arabic}

\section{Motivaci\'on y Prop\'osito}
\label{sec:motivation}

El nematodo C. elegans es un organismo ampliamente utilizado
y se ha convertido en un importante modelo de estudio para la
biolog�a, especialmente la gen�tica del desarrollo.
Este organismo presenta la ventaja de que todos los individuos son exactamente
iguales a nivel celular, poseen cortos ciclos de vidas y una r�pida
gen�tica. Por esta raz�n, se pueden detectar tipos salvajes de este organismo 
y los experimentos son menos costosos, en comparaci�n 
con organismos m�s complejos. Es el �nico animal del que se conoce
cada divisi�n celular, desde la fertilizaci�n del huevo hasta la etapa
adulta, as� como el diagrama completo de las conexiones de esas
c�lulas.\\

El C. elegans tiene muchas c�lulas con equivalentes humanos, lo que 
hace posible estudiar y comprender como se manifiestan ciertas
enfermedades y condiciones relacionadas, e.g. adicci�n a las drogas,
envejecimiento, disfunci�n de ciertas prote�nas, etc.
As� mismo, al ser peque�o y transparente, se presta bien
a una gran variedad de t�cnicas de cribado de alto rendimiento
(HTS). El HTS es un m�todo de experimentaci�n cient�fica que permite
conducir millones de pruebas gen�ticas, bioqu�micas o farmacol�gicas,
a trav\'es de la rob\'otica, software de control y procesamiento de datos,
dispositivos para el manejo de l\'iquidos y detectores sensitivos.
A trav�s de este proceso se pueden identificar r�pidamente componentes
activos, anticuerpos o genes que modelan procesos biomoleculares
particulares, tal como se indica en \cite{highthroughput}. \\
Diversos ganadores del premio Nobel de Medicina o Fisiolog�a
han centrado sus estudios en gusanos, y en particular C. elegans, 
tales como Brenner, Sulston y Horvitz (2002), Fire y Mello (2008),
as� como el ganador del Nobel de Qu�mica, Martin Chalfie (2008).\\

Antes de ser cuantificados, los gusanos deben ser identificados. Este
proceso deber�a ser autom�tico debido a que es muy trabajoso para
ser efectuado manualmente en un tiempo factible. Curiosamente, a 
pesar de la utilidad del C. elegans para manipulaciones gen�ticas,
su utilizaci�n en procesos de cribado de alto rendimiento se
ha visto tambi\'en limitado por la necesidad de ensayos manuales 
muy trabajosos.\\

Esto conlleva a la necesidad de m\'etodos m\'as r\'apidos y consistentes.
Por esta raz\'on, un programa de computadora que permita detectar individuos 
C. elegans en im\'agenes digitales, proveer\'ia una soluci\'on autom\'atica
para el problema de reconocimiento. Esto mejorar\'ia tanto la precisi\'on como 
el tiempo requerido para la identificaci\'on de los individuos, con respecto
a la identificaci\'on manual, permitiendo, a su vez, transformar 
las im\'agenes en informaci\'on manejable.\\

El presente estudio, se centra en el dise\~no e implementaci\'on de un 
algoritmo de procesamiento de im\'agenes para detectar 
gusanos C. elegans en im\'agenes de microscopio. Se provee,
as\'i mismo, una metodolog\'ia general de detecci\'on de gusanos.
La caracter\'istica m\'as relevante para la 
mayor\'ia de los experimentos con C. elegans es la forma del 
gusano, y en ocasiones tambi\'en la rotaci\'on y direcci\'on de la misma.
El enfoque que aqu\'i se presenta, busca identificar, exclusivamente, 
la forma de los gusanos. Se estudia, entonces, si es posible detectar 
y ajustar estas formas de manera automatizada, y si esto puede
alcanzarse m\'as r\'apidamente que a trav\'es de la identificaci\'on manual.\\

Se utilizan gusanos, en estado de larva, en placas de microtitulaci\'on. Las
larvas se cultivan en medio l\'iquido, lo que causa que el fondo de las im\'agenes
sea muy claro. No obstante, los gusanos se solapan con frecuencia.
La implementaci\'on es integrada a \emph{Endrov}, una arquitectura de extensiones
de c\'odigo abierto, dirigida al an\'alisis de im\'agenes y procesamiento de datos,
que fue desarrollada y es actualmente utilizada en el
Departamento de Biociencias y Nutrici\'on del Instituto Karolinksa, lugar donde
se desarrolla este proyecto.


\begin{itemize}
\item Objetivo General
  \begin{itemize}
  \item Dise\~nar e implementar una metodolog\'ia basada en procesamiento de
    im\'agenes para detectar gusanos en im\'agenes de microscopio.
  \end{itemize}
\end{itemize}
\begin{itemize}
\item Objetivos Espec\'ificos
  \begin{itemize}
  \item Dise\~nar un algoritmo de detecci\'on de gusanos que reciba como 
entrada im\'agenes de gusanos en cultivo l\'iquido y retorne la forma 
de los gusanos presentes.
    \begin{itemize}
    \item Revisar los antecedentes relevantes en t\'ecnicas de segmentaci\'on de im\'agenes
    \item Dise\~nar un descriptor de forma y un m\'etodo de rasterizaci\'on para representar gusanos en t\'erminos num\'ericos.
    \item Revisar los antecedentes en ajuste de formas y reconocimiento
      de objetos, y proponer un enfoque de detecci\'on.
    \end{itemize}
  \item Implementar el algoritmo de detecci\'on dise\~nado, integr\'andolo 
    a \emph{Endrov} como extensi\'on.
  \end{itemize}
\end{itemize}


\section{Antecedentes}

La utilizaci\'on del C. elegans en experimentos que involucran 
cribado gen\'etico y qu\'imico se ha incrementado r\'apida y notablemente. 
Esto ha dado pie al desarrollo de
m\'etodos automatizados para analizar su comportamiento, en experimentos 
conducidos sobre grupos de estos organismos, tal como se indica en \cite{automated}.
En el estudio mencionado, se dividen las estrategias existentes para an\'alisis automatizado 
del C. elegans en tres grandes grupos, de acuerdo a su
enfoque metodol\'ogico. Estos son: seguimiento del comportamiento general,
detecci\'on y medici\'on de comportamientos distintos, y medici\'on completa
del comportamiento utilizando grandes conjuntos de datos. Todas estas
estrategias incluyen una etapa fundamental, que se centra
en la detecci\'on de los gusanos en el conjunto de im\'agenes que se utilizan.
El enfoque de detecci\'on var\'ia de una estrategia a otra, pero, por lo general,
comprende los procesos siguientes: extracci\'on de los gusanos del fondo de la imagen (segmentaci\'on), 
\emph{skeletonization} de las formas extra\'idas y parametrizaci\'on 
del contorno de los gusanos.\\ 

La \emph{skeletonization} y subsecuente parametrizaci\'on, se han convertido en un 
m\'etodo est\'andar. Sin embargo, dado que las propiedades de la imagen tales como
iluminaci\'on, ruido y desorden (e.g. huevos y rastros de gusanos) pueden variar
fuertemente de una imagen a otra y debido a que la segmentaci\'on depende directamente del
contexto visual, los par\'ametros de este \'ultimo proceso resultan altamente variables.
Los m\'etodos de segmentaci\'on m\'as utilizados en im\'agenes de gusanos comprenden:
cerrado morfol\'ogico, llenado de agujeros, m\'etodo del valor umbral y sus 
combinaciones.\\

La parametrizaci\'on de gusanos, que consiste en la descripci\'on de formas de gusanos
en t\'erminos num\'ericos, determina la variedad de formas que pueden ser
obtenidas a trav\'es de la asignaci\'on de diferentes valores a los par\'ametros.
El enfoque m\'as com\'un se centra en definir par\'ametros que permitan
la reproducci\'on de una forma de gusano gen\'erica, normalizada para la 
posici\'on, orientaci\'on y escala de un esqueleto de gusano.\\

En \cite{automated}, se sostiene que entre aquellos programas que 
hacen seguimiento de m\'ultiples gusanos, muy pocos intentan resolver
el problema de solapamiento, que surge cuando dos o m\'as gusanos
se cruzan entre s\'i, o bien cuando gusanos individuales se enrollan,
lo que suele llevar a detecciones incorrectas o faltantes. 
Pese a que hay algoritmos que est\'an siendo desarrollados para resolver
este problema, tal como se indica en \cite{huang}, se sigue careciendo
de soluciones que permitan detectar la totalidad de los individuos
de forma autom\'atica.\\

Estudios muy recientes presentan nuevos enfoques para detectar gusanos 
individuales en agrupaciones enredadas (aquellos donde ocurre solapamiento).
Riklin Raviv et al. \cite{individual1} presentan un enfoque para extraer
objetos enredados, basado en sus propiedades morfol\'ogicas. Este estudio
aborda el problema de desenredar agrupaciones de C. elegans en experimentos
de cribado de alto rendimiento. Este m\'etodo se basa en conceptos de 
aprendizaje de m\'aquina y teor\'ia de grafos, y utiliza el esqueleto 
del gusano como descriptor de forma. Los segmentos de agrupaciones de gusanos
son representados como v\'ertices de grafo y se lleva a cabo una b\'usqueda
de los caminos de gusanos m\'as prometedores en el grafo. La detecci\'on
de los descriptores de forma m\'as prometedores dentro de la b\'usqueda
es guiada por una distribuci\'on de probabilidad, basada en el modelo
probabil\'istico presentado en \cite{individual2}.\\ 
Los enfoques presentados en \cite{individual1,individual2} corresponden
a estudios consecutivos y complementarios centrados en la detecci\'on de
gusanos individuales en im\'agenes digitales.
 Los resultados presentados indican un porcentaje de detecci\'on acertada 
de 89\% del total de la muestra, en promedio.
Es importante destacar que los dos estudios previamente mencionados fueron
desarrollados al mismo tiempo que el presente trabajo y con similares
fechas de finalizaci\'on, por lo que hab\'ia desconocimiento de su existencia. 
No obstante, el enfoque de detecci\'on y la metodolog\'ia presentada en este
trabajo, dista bastante de los estudios mencionados.\\


Existen entonces, diversos estudios en procesamiento de im\'agenes
y visi\'on artificial que tratan el an\'alisis automatizado del 
C. elegans y de nematodos en general. La mayor\'ia de estos 
estudios se centran en la locomoci\'on de gusanos, donde el proceso
de identificaci\'on y seguimiento es realizado a trav\'es del procesamiento
simult\'aneo de un conjunto de im\'agenes y no solo una. 
Se evidencian tres procesos fundamentales en las estrategias de 
detecci\'on como lo son: segmentaci\'on de la imagen, esqueletonizaci\'on y 
parametrizaci\'on de forma. El resto de los procesos involucrados en la
detecci\'on var\'ian dependiendo del enfoque, e involucran, en casi todos los casos,
el procesamiento de conjuntos de im\'agenes y no de im\'agenes individuales, 
como fue antes mencionado.\\

A pesar de que algunos m\'etodos automatizados de detecci\'on de gusanos
son capaces de detectar correctamente una gran parte de la muestra, pocos intentan
resolver el problema del solapamiento de gusanos y ninguno lo
soluciona exitosamente.


\section{Estructura del documento}
El presente documento est\'a divido de la siguiente forma:
\begin{itemize}
\item \textbf{Cap\'itulo 1: Introducci\'on}\\
  Se desarrolla la motivaci\'on y el prop\'osito de este estudio. Luego, se
  presentan los antecedentes en detecci\'on de gusanos, indicando los 
  diferentes enfoques, logros y dificultades persistentes.
\item \textbf{Cap\'itulo 2: Marco Te\'orico}\\
  Se abarca la teor\'ia relacionada con el problema y la soluci\'on
  planteada, destacando por t\'opico, los diferentes enfoques que han sido
  previamente estudiados.
\item \textbf{Cap\'itulo 3: Metodolog\'ia de la Soluci\'on}\\
  Se presenta la metodolog\'ia general de la soluci\'on.
  Primero, se desarrolla el razonamiento que sustenta la soluci\'on propuesta.
  Seguidamente, se explica cada etapa de la metodolog\'ia, justificando el enfoque
  escogido. Por etapa, se presentan los detalles de implementaci\'on 
  m\'as relevantes, que dan origen al algoritmo desarrollado en este trabajo.
\item \textbf{Cap\'itulo 4: Experimentos y Resultados}\\
  Se presentan los experimentos llevados a cabo para evaluar el rendimiento
  de la soluci\'on propuesta. El prop\'osito y caracter\'isticas de cada
  experimento son descritos. Luego, se presentan y discuten los resultados
  obtenidos.
\item \textbf{Cap\'itulo 5: Conclusiones y Observaciones Futuras}\\
  Las conclusiones del trabajo son presentadas, as\'i como algunas
  observaciones futuras.
\end{itemize}
