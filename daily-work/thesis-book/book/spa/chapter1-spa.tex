%\cleardoublepage
\thispagestyle{empty}
\chapter{Introduction}
\pagenumbering{arabic}

\section{Motivaci\'on y Prop\'osito}
\label{sec:motivation}

El nematodo C. elegans es un organismo ampliamente utilizado
y se ha convertido en un importante modelo de estudio para la
biolog�a, especialmente la gen�tica del desarrollo.
Presentan la ventaja de que todos los individuos son exactamente
iguales a nivel celular, poseen cortos ciclos de vidas y una r�pida
gen�tica. Por esta raz�n, tipos salvajes de este organismo pueden ser 
detectados y los experimentos son menos costosos en comparaci�n 
con organismos m�s complejos. Es el �nico animal del que se conoce
cada divisi�n celular, desde la fertilizaci�n del huevo hasta la etapa
adulta, as� como el diagrama completo de las conexiones de esas
c�lulas.\\

El C. elegans tiene muchas c�lulas con equivalentes humanos, lo que 
hace posible estudiar y comprender como se manifiestan ciertas
enfermedades y condiciones relacionadas, e.g. addicci�n a las drogas,
envejecimiento, disfunci�n de ciertas prote�nas, etc.
As� mismo, al ser peque�o y transparente, se presta bien
a una gran variedad de t�cnicas de cribado de alto rendimiento
(HTS). El HTS es un m�todo de experimentaci�n cient�fica que permite
conducir millones de pruebas gen�ticas, bioqu�micas o farmacol�gicas.
A trav�s de este proceso se pueden identificar r�pidamente componentes
activos, anticuerpos o genes que modelan procesos biomoleculares
particulares. [COLOCAR CITA Wikipedia High Throughput Screening]\\

Diversos ganadores del premio Nobel de Medicina o Fisiolog�a
han centrado sus estudios en gusanos, y en particular C. elegans, 
tales como Brenner, Sulston y Horvitz (2002), Fire y Mello (2008),
as� como el ganador del Nobel de Qu�mica, Martin Chalfie (2008).\\

Antes de ser cuantificados, los gusanos deben ser identificados. Este
proceso deber�a ser autom�tico debido a que es muy trabajoso para
ser efectuado manualmente en un tiempo factible. Curiosamente, a 
pesar de la utilidad del C. elegans para manipulaciones gen�ticas,
su utilizaci�n en procesos de cribado de alto rendimiento se
ha visto tambi\'en limitado por la necesidad de ensayos manuales 
muy trabajosos.\\

Esto conlleva a la necesidad de m\'etodos mas r\'apidos y consistentes.
Por esta raz\'on, un programa que permita detectar individuos 
C. elegans en im\'agenes digitales, proveer\'ia una soluci\'on autom\'atica
para el problema de reconocimiento. Esto mejorar\'ia tanto la precisi\'on como 
el tiempo requerido para la identificaci\'on de los individuos, con respecto
a la identificaci\'on manual, permitiendo, a su vez, transformar 
las im\'agenes en informaci\'on manejable.\\

El presente estudio, se centra en el dise\~no e implementaci\'on de un 
algoritmo de procesamiento de im\'agenes para detectar 
la forma de gusanos C. elegans en im\'agenes de microscopio. 
Las caracter\'istica mas relevante para la 
mayor\'ia de los experimentos con C. elegans es la forma o silueta del 
gusano, y en ocasiones tambi\'en la rotaci\'on y direcci\'on de la misma.
El enfoque que aqu\'i se presenta, busca identificar, exclusivamente, 
la forma de los gusanos. Se estudia, entonces, si es posible detectar 
y ajustar estas formas de manera automatizada, y si esto puede
alcanzarse m\'as rapidamente que a trav\'es de la identificaci\'on manual.\\

Se utilizan gusanos, en estado de larva, en placas de microtitulaci\'on. Las
larvas se cultivan en medio l\'iquido, lo que causa que el fondo de las im\'agenes
sea muy claro. No obstante, los gusanos se solapan con frecuencia.
La implementaci\'on es integrada a \emph{Endrov}, una arquitectura de extensiones
de c\'odigo abierto, dirigida al an\'alisis de im\'agenes y procesamiento de datos,
que fue desarrollada y es actualmente utilizada en el
Departamento de Biociencias y Nutrici\'on del Instituto Karolinksa, lugar donde
se desarrolla este proyecto.


\begin{itemize}
\item Objetivo General
  \begin{itemize}
  \item Dise\~nar e implementar un algoritmo de detecci\'on de gusanos
    en im\'agenes de microscopio.
  \end{itemize}
\end{itemize}
\begin{itemize}
\item Objetivos Espec\'ificos
  \begin{itemize}
  \item Dise\~nar un algoritmo de detecci\'on de gusanos que reciba como 
entrada im\'agenes de gusanos en cultivo l\'iquido y retorne la silueta 
de los gusanos presentes.
  \item 
    \begin{itemize}
    \item Revisar los antecedentes relevantes en t\'ecnicas de segmentaci\'on de im\'agenes
    \item Dise\~nar un descriptor de forma y un m\'etodo de rasterizaci\'on para representar gusanos en t\'erminos n\'umericos.
    \item Revisar los antecedentes en ajuste de formas y reconocimiento
      de objetos, y proponer una metodolog\'ia de detecci\'on.
    \end{itemize}
  \item Implementar el algoritmo de detecci\'on dise\~nado, integr\'andolo 
    a \emph{Endrov} como extensi\'on.
  \end{itemize}
\end{itemize}


\section{Earlier Work}

C. elegans has been deployed quickly in genetic screens and chemical
screens, as mentioned in \cite{automated}. In the previously mentioned paper,
the strategies for automated analysis of C. elegans locomotion are
divided in three groups according to their methodological approach:
Tracking overall behavior, detection and measurement of distinct behaviors and
measuring the complete behavioral repertoire using large parameters sets.
All of these strategies are based on an initial processing step that performs the detection of the worm shapes in the images. This usually involves the extraction of the worms from the background (segmentation), reduction to a skeleton and parameterization of worm outlines.

Reduction to skeleton and subsequent parameterization have become a standard method. 
However, since the image properties such as lighting, noise and clutters 
(e.g. worm tracks and eggs), can vary strongly from one image to another and 
the segmentation depends directly on the visual context, the  
parameters for this process are highly variable. The segmentation methods 
that are usually used 
on worm images are thresholding, morphological closing, 
hole filling, and their combinations.
It is stated in \cite{automated} that among those programs that track 
multiple worms, few attempt to resolve the problem that arise when 
worms interact or when individual worms coil up on themselves, which may
severely affect the individual worm identification. In \cite{huang} is indicated
that programs and algorithms are being developed to address this problem.\\

The  numerical description of a worm shape or worm parameterization 
determines the range of possible shapes a worm may adopt. Diverse 
representations have been used from one approach to another. The 
most common is the reproduction of an abstract shape, normalized for position, 
orientation, scale and parameterization of the 
worm skeleton.\\

Very recent studies present new approaches for detecting individual worms
in cluttered clusters. Riklin Raviv et al. \cite{individual1} present an approach for
 extracting cluttered objects based on their morphological properties. This study addresses
the problem of untangling C. elegans clusters in high-throughput screening experiments.
The method is based on concepts from machine learning and graph theory. The worm
skeleton is used as shape descriptor. 
The clustered worm segments are represented
as graph vertices and then a search for more likely worm paths in the graph 
is carried out. The detection of the most likely worm descriptors within the graph
search is guided by a probability distribution, defined by a low dimensional feature 
space that captures the worms' variability. This probabilistic shape model and a 
similar worm detection approach was first
presented by W\"{a}hlby et al. in \cite{individual2}.


There are many studies in computer vision dealing with the automated
analysis of C. elegans and nematodes in general. Most studies are
based on worm locomotion, so the process of identification and 
tracking is performed by the simultaneous analysis of a set of images, rather
than just one. There is a standard general strategy followed for an initial
worm identification step consisting in segmentation, skeleton reduction and
shape parameterization, while the matching strategies depend on the approach
and normally involves image sets and not individual images, as explained.
Although some automated worm detection approaches are able to identify
individual worms and group of worms, few of them attempt to overcome the 
problem of worm interaction and none solves it successfully.

\section{Structure of Thesis}
This document is divided the following way:
\begin{itemize}
\item \textbf{Chapter 1: Introduction}\\
  The motivation and purpose of the thesis is discussed. Then, earlier work
  on worms detection is presented, pointing out different approaches, achievements
  and current problems.
\item \textbf{Chapter 2: Theoretical Framework}\\
  The theory related to the problem and the solution is explained. Each topic is described,
  and the different studied approaches are addressed.
\item \textbf{Chapter 3: Methodology}\\
  The proposed solution is discussed and presented as a methodology. Then, each step 
  involving the methodology is explained thoroughly, comprising a description of the 
  approach and some implementation details.
\item \textbf{Chapter 4: Experiments and Results}\\
  A series of experiments to test the performance of the solution are presented. 
  The purpose and characteristics of the experiments are described. Then, the results are shown and discussed.
\item \textbf{Chapter 5: Conclusions and Future Work}\\
  The conclusions obtained from the discussion of the results are presented. 
Suggestions for future work are given.
\end{itemize}
