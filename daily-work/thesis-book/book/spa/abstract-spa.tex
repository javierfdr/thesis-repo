\begin{abstract}

   % What is C. elegans and why is it so important.
   % The problem: manual assays are to intensive, need
   % for automated solutions.
   % What do we provide: a image processing methodology that
   % provides semi-automatic solution, for Endrov (open source
   % image analysis software)
   % Few methods attempt to resolve cluttering. Individual
   % images are not usually addressed (large datasets).
   % Automated solutions fit 85 percent of the image. We 
   % fit 100% 

El nematodo C. elegans es un organismo ampliamente
utilizado. Posee muchas c\'elulas con equivalentes humanos y otras
condiciones especialmente favorables, que lo han convertido en modelo
de estudio para la biolog\'ia, especialmente la g\'enetica del
desarrollo. As\'i mismo, al ser peque\~no y transparente se presta bien
a una gran variedad de t\'ecnicas de cribado de alto rendimiento
(HTS). \\ 

La identificaci\'on de gusanos deber\'ia automatizarse lo mas
posible dado que es muy trabajoso efectuarla manualmente.
En este trabajo se presenta una metodolog\'ia de procesamiento de im\'agenes
para detectar C. elegans en im\'agenes obtenidas por microscop\'ia de 
alto rendimiento. La soluci\'on semi-autom\'atica que aqu\'i se provee, permite identificar
eficazmente gusanos particulares en agrupaciones de gusanos. En
t\'erminos generales, el proceso  consta de lo siguiente:
una imagen dada es segmentada, separando as\'i grupos de gusanos del
fondo de la imagen. Se detectan gusanos particulares de manera autom\'atica, siguiendo un
proceso de comparaci\'on y ajuste de siluetas de gusanos. Este proceso
se basa en encontrar siluetas factibles dentro de una agrupaci\'on, 
minimizando la distancia que existe entre dicha agrupaci\'on y 
siluetas gen\'ericas que son deformadas para ajustarse a ella.
Las conformaciones de gusanos ajustadas incorrectamente 
pueden ser corregidas f\'acilmente de manera manual.\\

La metodolog\'ia provista presenta un enfoque innovador para
detectar exitosamente gusanos C. elegans particulares en 
im\'agenes de microscopio. Los resultados muestran que esta soluci\'on
semi-autom\'atica permite detectar, correctamente, la silueta del 100\% de 
los gusanos presentes en una imagen determinada, a diferencia de otros m\'etodos
automatizados que alcanzan a detectar, en promedio, menos del 90\% ,
para conjuntos de pruebas similares.
Por lo general, el proceso es completado en menos de minuto y medio
para im\'agenes dif\'iciles. Para im\'agenes mas sencillas, los gusanos
puede ser identificados en su totalidad de manera enteramente 
autom\'atica. La precisi\'on de la detecci\'on y el tiempo requerido 
para calcularla son mejorados notablemente con respecto a la
idenficaci\'on manual.\\

La soluci\'on fue implementada en \emph{Java} e integrada a
\emph{Endrov}, una arquitectura de extensiones de c\'odigo abierto para
an\'alisis de imagenes, y ser\'a utilizada en el \emph{Departamento de
  Biociencias y Nutrici\'on, Instituto Karolinksa, Suecia}

\end{abstract}
