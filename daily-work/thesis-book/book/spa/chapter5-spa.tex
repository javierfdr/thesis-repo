%avoid page number on blank pages when cleared
\thispagestyle{empty}
\cleardoublepage
\chapter{CONCLUSIONES Y TRABAJOS FUTUROS}

En este cap\'itulo se presentan las conclusiones sobre el enfoque 
metodol\'ogico provisto y sobre los resultados
de la soluci\'on implementada.
Seguidamente, se presentan algunas sugerencias de
trabajo futuro, indicando las modificaciones que pueden llevarse 
a cabo para mejorar la soluci\'on.

\section{Conclusiones}

\subsection*{Metodolog\'ia de la Soluci\'on e Implementaci\'on}

La metodolog\'ia propuesta provee un enfoque semi-autom\'atico para 
detectar y ajustar la forma de gusanos individuales en im\'agenes
de microscopio. Esto permite convertir las im\'agenes de los gusanos
en informaci\'on manipulable y medible por computadora. La metodolog\'ia
es lo suficientemente general para ajustarse 
a diferentes tipos de im\'agenes y especies de gusanos, as\'i como
diferentes enfoques de implementaci\'on por etapa. \\
El algoritmo desarrollado, que se deriva de la implementaci\'on provista,
es lo suficientemente eficaz como para proveer la detecci\'on de 
la totalidad de los gusanos en corto tiempo utilizando una computadora
personal, mejorando as\'i mismo el tiempo requerido y la precisi\'on
de la detecci\'on manual.
Este estudio constituye uno de los primeros trabajos que trata la detecci\'on de gusanos
en im\'agenes individuales (solo se conoce otro estudio que fue 
desarrollado al mismo tiempo) y el primero que permite que detectar la totalidad de los
gusanos en las im\'agenes.\\

Se dice que la soluci\'on es semi-autom\'atica por la necesidad de 
realizar ajustes manuales en dos etapas del proceso: detecci\'on de
puntos extremos y ajuste final de formas. Una vez que los puntos extremos
son completamente identificados, la parte autom\'atica de la soluci\'on
provee un alto porcentaje de acierto (m\'as de tres cuartas partes
del total de los gusanos, en el peor caso), que puede incluso ser 
\'optimo en imagenes \emph{f\'aciles}. El proceso de ajuste manual
de conformaciones permite al usuario corregir las fallas de detecci\'on de la
soluci\'on autom\'atica, proveyendo as\'i un ajuste \'optimo o total.\\

La etapa de \emph{procesamiento inicial} es ejecutada muy r\'apidamente,
constituyendo menos del $1\%$ del tiempo total de ejecuci\'on. Por otro
lado, la etapa de optimizaci\'on de ajuste de formas corresponde al
proceso mas demandante, consumiendo mas del $99\%$ del tiempo total de
ejecuci\'on, excluyendo el tiempo requerido para correcciones manuales.\\

La soluci\'on implementada fue integrada con \'exito a \emph{Endrov} como
extensi\'on, y esta siendo utilizada en este momento en los laboratorios
del Departamento de Biociencias y Nutrici\'on del Instituto Karolinska, en
Fleminsberg, Suecia.



\subsection*{Detecci\'on y Ajuste en Gusanos Aislados y Agrupaciones}

La soluci\'on implementada provee la detecci\'on completa de todos
los gusanos aislados, a trav\'es de un proceso autom\'atico y sin 
requerir adici\'on manual de puntos extremos o correcci\'on
de ajuste. A partir de los gusanos aislados se puede calcular 
exitosamente un perfil de los gusanos en la imagen.\\

La detecci\'on y ajuste de formas de gusanos en agrupaciones represent\'o
el proceso mas desafiante de la metodolog\'ia. Una alta densidad
de gusanos lleva a m\'ultiples solapamientos y a la creaci\'on
de agrupaciones, donde algunos los puntos extremos pueden no
ser detectados. Algunos gusanos no se pueden detectar correctamente
si hay puntos extremos faltantes, por lo que requieren ajustes manuales.\\

La forma de gusanos aislados puede ser descrita con precisi\'on en todos
los casos. Las formas ajustadas en agrupaciones de gusanos representan
en todos los casos de una forma muy precisa a los gusanos reales en la imagen.

\subsection*{Predicci\'on de Caminos}

La heur\'istica de predicci\'on de caminos permite mejorar considerablemente
la efectividad de detecci\'on y ajuste del proceso autom\'atico. Debido
a que las agrupaciones de gusanos proveen una gran cantidad de caminos
posibles entre pares de puntos extremos, la predicci\'on de caminos se 
convierte en una herramienta \'util para determinar los caminos m\'as
probables que parten de cada punto extremo. Sin embargo, dada la naturaleza
altamente deformable de las formas de gusanos, la heur\'istica falla
en ocasiones en determinar el camino correcto para algunos puntos
extremos.

\subsection*{Optimizaci\'on y Funci\'on de Energ\'ia}

El proceso de optimizaci\'on permite reducir considerablemente la diferencia
entre la silueta gen\'erica construida en base al descriptor de forma y 
la imagen, a trav\'es de la deformaci\'on.
El enfoque de b\'usqueda local para el m\'etodo de optimizaci\'on 
result\'o ser efectivo y r\'apido en la obtenci\'on de ajustes precisos.
La eficacia de la b\'usqueda local reside en el hecho de que la silueta
original es construida sobre un camino del esqueleto, y dado que este
tiende al eje medio de los objetos en la imagen, la silueta tiende a 
estar situada cerca de un gusano real en la imagen.\\

La funci\'on objetivo, formulada en t\'erminos de energ\'ia, es lo 
suficientemente expresiva para posicionar la conformaci\'on correcta
entre las dos mejores conformaciones posbiles por punto extremo. En
la gran mayor\'ia de los casos la conformaci\'on correcta resulta
tener el menor valor de energ\'ia, conduciendo por lo tanto a un ajuste correcto.
Sin embargo, las dos mejores conformaciones tienden a estar muy cerca la una de la otra,
en t\'erminos de energ\'ia, lo que lleva a errores de ajuste. Esto hace que
la funci\'on objetivo no sea lo suficientemente expresiva como para proveer
un ajuste autom\'atico perfecto en im\'agenes dif\'iciles.

\section{Trabajos Futuros}

A continuaci\'on, se presentan una serie de sugerencias de trabajos
futuros para mejorar la soluci\'on provista en este estudio.

\subsection*{Funci\'on de Energ\'ia}

Una funci\'on de energ\'ia mas sofisticada podr\'ia permitir obtener diferencias
mas grandes entre las conformaciones optimizadas, conduciendo a un porcentaje
de detecci\'on mas alto. Una posible formulaci\'on podr\'ia consistir en utilizar
el mapa de distancias para evaluar la cercan\'ia de la silueta que se deforma
a contornos en la imagen, y empujarla hacia ellos. Los p\'ixeles de fondo
(de valor $0$ en el mapa de distancia) tendr\'ian que ser penalizados de cierta
forma, y deber\'a buscar un equilibrio de distancias en la suma total del \'area
de la silueta.

\subsection*{Detecci\'on de Puntos Extremos}
Dado que la detecci\'on de puntos extremos juega un papel fundamental
en el proceso de detecci\'on, una t\'ecnica m\'as elaborada para la 
identificaci\'on de puntos extremos permitir\'ia mejorar la eficiencia de la
soluci\'on autom\'atica, reduciendo o eliminando la necesidad de agregar
puntos manualmente.\\

Una forma de encontrar puntos extremos faltantes podr\'ia consistir en sacar
provecho de la b\'usqueda informada que efect\'ua el algoritmo de predicci\'on
de caminos. La idea consistir\'ia en trazar caminos a partir de los puntos extremos,
hasta alcanzar un longitud de camino fija, \emph{e.g.} la longitud estimada
de los gusanos en la imagen. Una vez que se alcanzara este punto, si no se han
encontrado puntos extremos, se agregar\'ia uno en esta posici\'on. Este enfoque
tiene el problema de que se pueden agregar puntos extremos incorrectos. En vista
de esto, se podr\'ia ejecutar inicialmente el algoritmo de ajuste de formas para
tener un visi\'on previa del \'area que ha sido cubierta, y seguidamente, llevar
a cabo un b\'usqueda de puntos extremos posibles en las \'areas de la imagen
que no fueron cubiertas.\\

Otra soluci\'on es la de considerar las intersecciones entre los caminos
del esqueleto como posibles puntos extremos, y realizar un an\'alisis mas
profundo sobre la factibilidad de estos puntos y de las conformaciones 
obtenibles a partir de estos.

\subsection*{Seguimiento de Movimiento de Gusanos}
Una vez que los gusanos en la imagen son detectados por completo, se 
tiene informaci\'on acerca de su posici\'on y tama\~no, y as\'i mismo es
posible calcular informaci\'on adicional tal como la rotaci\'on del gusano y la
posici\'on de la cabeza y la cola. Este tipo de informaci\'on 
podr\'ia ser muy valiosa para los algoritmos de seguimiento de movimiento
de gusanos en im\'agenes, y otros enfoques basados en an\'alisis simult\'aneo
de grandes conjuntos de datos (tales como aquellos que son cubiertos en
\cite{automated}), para detectar gusanos en agrupaciones.

\subsection*{Desarrollo Actual}

Actualmente, el autor de este trabajo y el estudiante de Doctorado Johan Henriksson,
est\'an trabajando en la automatizaci\'on de la soluci\'on provista, as\'i como
en el refinamiento de la implementaci\'on.\\
Recientemente se consigui\'o reducir el tiempo de detecci\'on autom\'atica en mas
del triple, al refinar el algoritmo de predicci\'on de caminos y reducir la cantidad
de conformaciones generadas.\\

Por otro lado, se est\'a utilizando la implementaci\'on integrada a \emph{Endrov}
del rasterizador de pol\'igonos y descriptor general de formas para desarrollar
un algoritmo de rastreo de peces en video.








