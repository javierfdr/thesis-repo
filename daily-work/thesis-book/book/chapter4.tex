\cleardoublepage
\chapter{Experiments and Results}
\label{chap:experiments}

%Introduce Chapter: 3 experiments (test set size = 3) to test the whole
%shape matching methodology. Its separated in three main processes: initial
%processing, automatic shape matching and manual fixing. The results are 
%presented and discussed.

In this chapter the different experiments carried out to test the perfomance
of the shape matching solution are 
presented, explaining the reason for their choice and the test objectives.
Then the results are presented and discussed, pointing out the advantages and drawbacks of the suggested
shape matching model, that are infered from the first ones.

\section{Experiments}
\label{sec:experiments}

In order to test the implemented solution, three digital images of worms in liquid
media where selected from the image database of the Department of Bioscience and Nutrition
of the \emph{Karolinska Institute}. Each image corresponds to a different difficulty level, 
following the criteria of number of worms and degree of overlap. Thus the greater the number 
of worms in the image and the greater the number of worms that are connected by overlapping,
the greater the difficulty level. The overlapping degree is determined by the number
of worms that belong to the different \emph{worm clusters} (first defined in Sec. \ref{sec:reasoning})
and the number of \emph{worm clusters} in the image.
The selected images are named \emph{test image 1}, \emph{test image 2} and \emph{test image 3}
and they are disposed in an increasing difficulty level order, from the easiest to the hardest. 
Below in Table \ref{tab:testset} the characteristics of the test set are presented.


\begin{table}[h]
  \caption{Test set images characteristics}
\begin{center}
\begin{tabular}[h]{|>{\columncolor[gray]{0.9}} p{2cm} |p{3cm} | p{2.8cm} | p{3cm}| p{2.8cm} |}
    \hline
    \rowcolor[gray]{.9}
    Test Image& Number of Isolated Worms & Number of Worm Clusters & Number of Worms per Clusters  & Total Number of Worms\\
    \hline
    Test 1 & 11/19 (57.8\%) & 3 & 8/19 (42.1\%) & 19 \\
    \hline
    Test 2 & 13/38 (34.2\%)& 5 & 25/38 (65.7\%) & 38 \\
    \hline
    Test 3 & 8/33 (24.2\%) & 3 & 25/33 (75.7\%)& 33 \\
    \hline
  \end{tabular}
\end{center}
  \label{tab:testset}
\end{table}


For each test image a series of experiments were carried out to test the performance of
the different processes that are involved in the solution methodology. The entire 
shape fitting process was divided into three stages that represent the 
main fitting steps. These are: \emph{initial processing}, \emph{automatic shape matching} and
\emph{matching manual fixing}.\\

The \emph{initial processing} involves all the image processing steps that are performed 
before the shape matching optimization process, as explained in Sec.\ref{sec:solmet},
such as: thresholding, distance transformation, skeletonization, 
clustering, endpoint detection and worm profiling. Among these, the distance transformation
and skeletonization follow already implemented and tested algorithms (covered in Sec.\ref{sec:metdt}
and Sec.\ref{sec:metsk}) and produce straight forward results, so there is no need to make further analysis
about them.
The clustering process follows the skeletonization process and is straight forward as well.
On the other hand the thresholding, endpoint detection and worm profiling processess vary 
from image to image, so different experiments are carried out in order to test these subprocesses
for the test set.\\

The \emph{automatic shape matching} stage consists in experiments dealing with the automatic
optimization process to match \emph{worm clusters} and contour following technique to match 
\emph{isolated worms} that produce a first fitting guess for the worms in the image. These 
experiments attempt to measure the efficacy and time efficiency of different variations of the
matching algorithm, in order to conclude about properties of the algorithm and the feasibility 
of the automatic solution.\\
The third stage,\emph{matching manual fixing}, attempts to measure the type and amount of 
manual modification operations that must be performed by the user in order to fix the 
matchings that were wrongly assigned by the automatic process. The experiment is also 
aimed to determine how much can the efficacy of the automatic solution be improved, in
order to get a better matching. Once the best possible matching is found through manual fixing,
a distribution of the matching energies is presented, that aimes to study the sensitivy of the
matching algorithm. This consist on a graph showing the best three conformations, in terms of the objective
function, starting from every endpoint, comparing how far are the top two conformations from
the right one.\\

For the first stage of \emph{initial processing} the results for the three test images are presented
at the same time, while for the stages of \emph{automatic shape matching} and \emph{matching manual fixing}
the result for the two stages are presented together for every individual test, since they are 
extremely related.


\section{Results}
\label{sec:results}

In this section the results obtained for the test set are presented and dicussed, discriminating between
the, previously defined, three main stages: \emph{initial processing}, \emph{automatic shape matching} and
\emph{matching manual fixing}. In the section \emph{initial processing} the results for the three different 
images in the test set are presented. On the other hand the section \emph{Shape fitting} presents the different
results for the stages \emph{automatic shape matching} and \emph{matching manual fixing} for every test image
individually.

\subsection{Initial Processing}
\label{sec:initproc}

This section presents the results for the experiments carried out over the test set for the 
processes: thresholding and endpoint finding. It is also present a brief discussion about
the result for the worm profiling process.

\subsubsection*{Thresholding}

As explained in Sec.\ref{sec:thresimp} \emph{Endrov} has implementations for the 
thresholding filters \emph{Fukunaga},\emph{Max entropy}, \emph{Otsu} and \emph{Percentile}.
The process to determine that a binary image represents sufficiently well the original 
image was done by visual appreciation. In order to select an appropriate thresholding
filter, the four previously mentioned were tried on every test image, tweaking the
input parameters until the best possible binary image was obtained from every one.
It resulted that among the four methods, the \emph{Percentile} filter generated the
best binary images for the three tests. The next closer method was \emph{Fukunaga}
that produced acceptable solutions after the combination of binary images
generated from different number of classes, however the results were not as good as
the ones obtained with \emph{Percentile}, requiring more processing as well.\\

Following this experiment the \emph{Percentile} method was selected by being considered
sufficiently good to generate a binary representation for the worms in the image 
and easy to manipulate. Below in table \ref{tab:threshold} the best percentile 
values for every test image are presented.


\begin{table}[h]
  \caption{Best percentile value for Percentile Thresholding for test image}
\begin{center}
\begin{tabular}[h]{|>{\columncolor[gray]{0.9}} c |c|c|c|}
    \rowcolor[gray]{.9}
    \hline
    Test & Test Image 1 & Test Image 2 & Test Image 3\\
    \hline
    Percentile Value & 0.074 & 0.1 & 0.11\\
    \hline
  \end{tabular}
\end{center}
  \label{tab:threshold}
\end{table}

The resultant best percentile values oscillate between $0.074$ and $0.11$, and are 
simple to determine using \emph{Endrov}.   

\subsubsection*{Endpoint detection}

Once the process of skeletonization is performed the worm endpoints are detected
in the image by finding the extremes skeleton points. Starting from these the 
skeleton is expanded following the directional neighbor process explained in
Sec.\ref{sec:wend} until contour points are found. These contour points will
correspond to worms endpoints.\\
As explained, worm endpoints are detected by first identifying extremes in the worm
skeleton. However when the extreme of a given worm in an image overlaps with 
another worm shape, the binary image will consider this worms as part of the 
same shape and in consequence the shape skeleton will generate a continuous 
path and not an extreme skeleton point, thus the endpoint will not be detected.\\
Since the solution methodology is centered on finding paths between endpoints and 
then determining the most likely worm paths, having missing endpoints would
affect greatly the matching accuracy, as it is shown later in the subsections 
\emph{automatic shape matching} of every test image results.\\

Since the worm endpoints are simple to determine visually, a manual process
of adding and removing pixel operations can be performed fastly in order to
add the missing endpoints and disconnect them from the rest of skeleton, as
explained in Sec.\ref{sec:endpointop}.
The removing operation is required because by definition a worm endpoint 
belongs to a extreme of the skeleton and then only one directional 
path should be able to start from it, \emph{i.e.}a worm endpoint has only
one \emph{cardinal neighbor} belonging to the skeleton.
As described in Sec.\ref{sec:endpointop},
the process consists basically on selecting a skeleton pixel that is not marked as
endpoint and then selecting the wrong skeleton neighbor pixel to disconect the path 
from the wrong part of the skeleton. \\

Below, Table \ref{tab:endtable} presents the number of endpoints identified
automatically for ever test image and the number of operations that must be
performed to fix the missing endpoints. An operation corresponds to the combination
of an add/remove pixel operation.


\begin{table}[h]
  \caption{Worm endpoints detection and fixing for the test set}
\begin{center}
\begin{tabular}[h]{|>{\columncolor[gray]{0.9}} p{2cm} |p{1.9cm}|p{2cm}|p{2.2cm}|}
    \rowcolor[gray]{.9}
    \hline
    Test Image & Total Endpoints & Detected Endpoints & Add/Remove Operations\\
    \hline
    Image 1 & 38 & 38 (100\%) & 0 \\
    \hline 
    Image 2 & 76 & 57 (75\%) & 19 \\
    \hline 
    Image 3 & 66 & 53 (80\%) & 13 \\
    \hline
  \end{tabular}
\end{center}
  \label{tab:endtable}
\end{table}

From the table above, it can be infered that a high number of worms belonging
to \emph{worm clusters} increases the chances to have endpoints overlapping, 
as well as it does a higher number of worms in total in the same space.
Considering the high amount of worms that belong to worm clusters 
for images one and two ($65\%$ and $75\%$ respectively) and the few
amount of clusters for each one (which increases the overlapping) the amount
of missing endpoints can be considered faily low, and is feasible in time 
to fix them manually.

\subsubsection*{Worm Profiling}

In order to generate an accurate worm profile of the worms present in the image
it is necessary to have \emph{isolated worms} in the processing image. The percentage 
of \emph{isolated worms} for every image were $57.8\%$, $34.2\%$ and $24.2\%$ respectively,
oscillating between $8$ and $13$ worms, as shown in Table \ref{tab:testset}. 
For all of this images the generated worm profiles were sufficiently accurate to
conduct the optimization shape matching process, whose results are presented in 
the subsections named \emph{automatic shape matching} of every test image results.

