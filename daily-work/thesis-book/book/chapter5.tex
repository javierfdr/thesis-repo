%avoid page number on blank pages when cleared
\thispagestyle{empty}
\cleardoublepage
\chapter{Conclusions and Future Work}

In this chapter, the conclusions obtained from the experiment results of the 
worm shape fitting methodology developed in this work are addressed.
Then some future work suggestions are presented, pointing out the modifications
that can be performed to improve the solution methodology.

\section{Conclusions}

\subsection*{Solution Methodology}

The proposed methodology provides a feasible semi-automatic solution
to identify and fit worm shapes in bright-field digital images. This allows
to turn the microscope worm images into manageable computer data and 
improve considerably the time cost and matching accuracy with respect
to manual identification.\\
The methodology design and implementation are efficient enough to provide
a complete identification of worms in short time using a personal computer.\\
The solution is said to be semi-automatic because of the need of manual
tweaking in two steps of the process: endpoint identification and
final matching. Once the endpoints are completely identified, the automatic
solution provides a high matching percentage (more than three quarters of 
the total, in the worst case), that can be even be optimal in \emph{easy} 
images. The final matching process allows the user to re-arrange the 
automatic solution, providing an optimal match for the image.\\
The \emph{Initial Processing} step is quickly executed, less than
$1\%$ of the total execution time. On the other hand The \emph{Shape Matching}
step is the most time consuming process, consuming the $99\%$ of the total
time excluding the time taken for manual adjustments (normally fast).

\subsection*{Isolated Worms and Worm Cluster matching}

The \emph{isolated worms} are fully identified in every image following
an automatic process, without requiring manual addition of endpoints
or match fixing. An accurate image worm profile can be successfully 
calculated from the \emph{isolated worms}.\\
The identification of worms in \emph{worm clusters} represented the most 
challenging process of the worm fitting methodology. A high density
of worms leads to multiple overlaps and clustering, where endpoints can fail
to be detected. Worms will not be identified correctly if endpoints are
missing, so manual adjustments are required.\\
The shape of \emph{isolated worms} can be traced accurately in all the cases.
For worms in clusters the shape is initially matched through the perturbation
of a descriptor that is built from a worm profile. Once the optimal shape
is found the contours of the generated shape are expanded or contracted
to fit the individual shapes. This process is performed successfully, and in
all the cases the best shape between endpoints is either exact or very close
to the worm shape in the image.

\subsection*{Path Guessing}

The path guessing heuristic improves considerably the matching accuracy
of the automatic process. Since worm clusters provide a large amount of 
candidate paths between pairs of endpoints, the path guessing
becomes a useful tool to determine the more likely worm paths departing
from every endpoint. Nonetheless, given the highly deformable nature
of worm shapes, the path guessing heuristic fails in occasions to determine
the correct path for endpoints.

\subsection*{Optimization and Energy Function}
The optimization process manages to reduce considerably the difference
between a descriptor based shape
and the matched worm shape, by the perturbation process performed over
the angles between control points. \\
The best-in-neighborhood local search approach for the optimization algorithm
is then an effective and fast way to obtain an accurate matching
shape by deformation. The efficacy of local search for this approach resides
in the fact that the original shape is constructed over a sub-path of the
skeleton image. Since the skeleton is an approximated medial axis path, 
the initial shape tends to be near (in shape distance terms) to the shape 
in the image.\\

The energy formulation is sensitive enough to position the optimal solution
among the two best possible conformations for every worm. For the vast majority
of the cases the optimal solution is considered to have the lowest 
energy value, hence
leading to a correct match. However, the top two conformations tend to be
very close to each other, which leads to matching errors, thus making
the objective function not sensitive enough to provide an automatic perfect
match in difficult images. A more sophisticated energy formulation, which
takes more advantage of the information in the image, could
differentiate better between conformations and possibly
lead to perfect automatic matches for difficult images.


\section{Future Work}

Below, a set of suggestions are given to improve the presented solution
methodology and to solve new problems based in this formulation.


\subsection*{Energy Formulation}
A more sophisticated energy formulation, defining the distance between 
shapes, would allow to have bigger differences between optimal conformations,
thus leading to a better matching. In this work, the external energy was
defined as the percentage of background pixels covered by the matching shape,
over the total area, while the internal energy was supposed to be stable,
provided the worm profile based shape construction. 
A better energy formulation would push the shape up to the contours of
worms in the image, and avoid to stabilize in worm intersections.\\
A possible better formulation could make use of the previously calculated
distance map, where there is a value assigned to every pixel
indicating the distance to the nearest background point. 
Then, in order to reduce the energy, the shape will tend to reach border pixels.
Since the distance map assigned distance value $0$ to background pixels, the 
presence of these would have to be penalized in some way.


\subsection*{Endpoint Detection}
Since the detection of endpoints play a fundamental role in the matching 
process, a more sophisticated detection technique would improve the 
efficiency of the automatic solution, and reduce the time spent in
manual endpoint addition. A suggested way of finding missing endpoints is
to use the path guessing algorithm to trace the more likely worm paths 
from endpoints, after a number of pixels have been covered (\emph{e.g.} the
estimated worm length calculated for profiling). Once this point is reached
a neighborhood analysis can be performed to look for pre-existing 
worm endpoints and then place a new one at this point in case no endpoint 
is found. This approach has the possible drawback that the path guessing
algorithm not always follows a correct path so incorrect endpoints would
be added. A way of reducing the misplacing of endpoints would be to execute
the matching algorithm with the missing endpoints, and then execute the 
endpoint finding algorithm covering skeleton path portions that were not 
matched previously.

\subsection*{Non-bipartite assignment}
In this work an improved greedy algorithm is used to find the best
possible set of matching shapes assignments between endpoints in order
to provide one and only one shape for endpoint and to cover the greatest
amount of endpoints as possible, thus maximizing the match. This algorithm
does not selected the best possible set of assignments in all the cases. 
An implementation of a non-bipartite graph assignment algorithm, where
the departure and arrival set are conformed by all the endpoints of a worm
cluster, would allow to obtain the best possible assignation, leading
probably to a higher matching percentage for the automatic algorithm.\\

It is however not clear if a different algorithm will improve precision.
Solutions for other problems not found by greedy algorithms tend to be
non-intuitive and physically infeasible.


\subsection*{Worm Movement Tracking}
Once the worms in an image are completely matched, information about their
position and size is obtained, and other extra information like rotation
and head-tail positions could be calculated. This kind of information from a
total match could be valuable for \emph{worm trackers} and other approaches 
based on large image datasets
(such as those reviewed in \cite{automated}) to untangle worm clusters and 
improve the matching accuracy.
