\documentclass[12pt,letterpaper,oneside]{book}

\RequirePackage{amsmath}

%compile just these 'include' tags
\includeonly{chapter1,chapter2,chapter3,chapter4,chapter5} 

\title{Presentation}

\begin{document}
\section*{Introduction}

\begin{itemize}
\item C.elegans: how are them, why are they being used.

High-throughput screening (HTS) is a method for scientific experimentation especially
 used in drug discovery and relevant to the fields of biology and chemistry. Using robotics,
 data processing and control software, liquid handling devices, and sensitive detectors, 
High-Throughput
 Screening or HTS allows a researcher to quickly conduct millions of biochemical, genetic or
 pharmacological tests. 

The roundworm, C.elegans, is a widely used organism. It has become
an effective model system for biological processes such as 
inmunity, behavior and metabolism.

HTS of C.elegans are now being used to test tens of thousands of chemical or genetic
perturbations to identify promising drug compounds and regulators
of disease

\item Problem: too many to fit them manually -> automated solution
Many scientific questions
require measurements on individual worms. Before quantification the
worm must be identified. This should be automatically since there are too many
worms to feasibly do this manually.

The deployment in high-throughput screens has been lim-
ited by labour-intesive manual assays used to score phenotypes, thus making arise an urgent
need for more rapid and more consistent methods.


\item Endrov: Particularly at the karolinska institute they will
  like it. Endrov intents to be next standard
\end{itemize}


%BACKGROUND

However, these methods exploit motion
cues to disambiguate touching worms and cannot be used
in still images of high-throughput screens.

The detection usually involves extraction of
the worm from the bacgkround (segmentation), reduction to skeleton and parameterization
of worm outlines.

 The numerical description of a worm shape or worm parameterization determines the range
of possible shapes a worm may adopt. Diverse representations have been used from one ap-
proach to another, being the most common: reproduction of an abstract shape normalized
for position, orientation, scale or any geometrical parameter and parameterization of the
worm skeleton.

  Endrov is an open source plugin architecture aimed for image analysis and data process-
ing. Endrov aimes to improve the features of the standard open source image anaylisis
program, ImageJ, by providing a more modern design.


  Endrov was developed at the TBU Group, Karolinska Institute and was officially released
on 17 June 2009, under BSD license.


\subsection*{Objectives}
\begin{itemize}
\item General Objective
\item Specific Objectives
\end{itemize}

\section*{Background}
\begin{itemize}
\item Tracking
\item Not tracking
\item Cluster problems. Trackers attempt to solve it through
  large data sets. Not trackers with sets of one image. High
  percentage, never total.
\item Segm-Sk-Parameterization
\item Optimization varies
\end{itemize}

\section*{Methodology}

\subsection*{Reasoning}
%reasoning Big Image
\begin{itemize}
\item Image must be segmented in order to separate the individual
  worms as much as possible. The image can be divided into object
  pixels and background pixels.
\item The endpoints of the worms must be detected/identified. A skeleton
  is calculated that traces a thin path between endpoints.
\item We separate the worms into two classes: isolated worms, and 
  worm clusters.
\item The shape of the isolated worms can be identified easily.
\item On the other hand, worm clusters require a matching process
  in order to fit the clustered worms.
  \begin{itemize}
  \item This requires a shape descriptor. 
  \item Worm Shape Rasterization
  \item And match optimization method
    \item The optimization method will consist in constructing a generic
      shape given a profile and a worm path, and deforming that shape
      until a best match is found.
  \end{itemize}
\item Incorrect matched can be fixed manually

\end{itemize}

\subsection*{Individual Explanation}

\begin{itemize}
\item First a thresholding algorithm calculates a binary image dividing
  in object and background pixels. The object pixels are those belonging
  to a worm shape.
  Noise is removed as much as possible.
\item A distance map is calculated that allow us to obtain the shortest
  distance to the background. Is used as a tool
  to trace contour, automatic generation of shape
  descriptors and skeletonzation, path guessing algorithm.
\item The skeleton is calculated. This is a connected thin path that tends
to represent the medial axis. Endpoints can be calculated as the extreme 
of the worms. 
\item Division isolated and cluster worms.
\item Isolated: easily traceable. Finding the 
closest contour pixel and following it.
\item Cluster: Construct a generic worm shape around
a medial axis and deform it until the more likely
worm shape is found.\\ 
For that we need a shape descriptor, a rasterization
method and distance measure between the matching 
model and the image.
\end{itemize}

\begin{itemize}
\item Shape descriptor: A profile can be calculated
from isolated worms.\\
A series of even distance control points are
calculated from the skeleton path. Given the profile
the contour points are calculated by finding the 
intersection between the bisectors and the 
thickness circle. A cardinal spline is used to 
trace the contour.
\item Rasterization is done with ear clipping and
triangle rasterization.
\item Optimization
  \begin{itemize}
  \item Deformation
  \item Energy formulation.
  \item Conformations between every logical 
    pair of endpoints
  \item Assignment algorithm
  \end{itemize}
\end{itemize}

\begin{itemize}
\item Manual Manipulation: Wrong conformations
  can be fixed
\item Endpoints as well.
\end{itemize}



\end{document}

