\documentclass[12pt]{article}
\usepackage{amsmath}
\title{\LaTeX}
\date{}
\begin{document}
  \maketitle 
  \begin{abstract}

   % What is C.elegans and why is it so important.
   % The problem: manual assays are to intensive, need
   % for automated solutions.
   % What do we provide: a image processing methodology that
   % provides semi-automatic solution, for Endrov (open source
   % image analysis software)
   % Few methods attempt to resolve cluttering. Individual
   % images are not usually addressed (large datasets).
   % Automated solutions fit 85 percent of the image. We 
   % fit 100% 


The nematode C.elegans is a widely used model organism. It has many cells with
human equivalents, making it possible to study pathways conserved in humans
and related conditions. Being small and transparent it also lends itself well to
a variety of high-throughput screening techniques.
Worm identification should be as automated
as possible since is too labor-intense and time-consuming to do it manually.\\

Here we present an image processing methodology to detect 
C.elegans in high-throughput microscope images. 
The provided semi-automatic solution makes it possible to effectively identify individual 
worms in worm clusters. In general terms the process is as follows: 
A given image is segmented, thus 
separating groups of worms from the background. Individual worms are detected
automatically, following a worm-shape matching process. For worm clusters, the 
matching process
is based on finding feasible worm shapes by minimizing the distance between
the cluster and generic worm shapes, that are deformed to fit it.
Wrong and missing conformations can be quickly fixed manually. \\

The provided methodology is a novel approach 
to successfully detect individual C.elegans worms in high-throughput microscope images.
Results show that this semi-automatic solution makes it possible to fit the shape of
100\% of worms in the image, unlike previous automated methods that usually reach less 
than 90\% in average, for similar test sets.
The detection process is usually achieved in less than a minute for difficult images. 
For easier images the total match can often be calculated in a fully automatic way.
Time cost and matching accuracy are considerably improved 
with respect to manual identification

The solution was implemented and adjusted to \emph{Endrov}, an open source
plugin architecture for image analysis, and is to be used at 
\emph{Department of Bioscience and Nutrition, Karolinska Institute, Sweden}. 
  \end{abstract}
\end{document}

%
%is based on finding feasible worm shapes by minimization of distance between the 
%any feasible worm shape in the cluster
%and a corresponding generic shape, that is deformed to fit it. 
