%\cleardoublepage
\thispagestyle{empty}
\chapter{Introduction}
\pagenumbering{arabic}

\section{Motivation and Purpose}
\label{sec:motivation}

C. elegans is a widely used model organism. It has the advantage of all worms 
being exactly the same down to the cellular level, short life cycles and rapid
 genetics. Thus, small deviations from wild-type can be detected and experiments
 are cheap compared to higher organisms. Nematodes have many cells with human
 equivalents making it possible to study many pathways conserved in humans, 
and related conditions e.g. drug addiction, aging, dysfunction of certain
 proteins etc. Being small and transparent it also lends itself well to a 
variety of microscopy-based high-throughput screening techniques.\\
 
Before quantification, the worms have to be identified; this should be 
automatic since there are too many worms to feasibly do this manually.
Despite the utility of C. elegans models for genetic 
manipulations, the deployment in high-throughput screens has been limited by 
labor-intensive manual assays used to score phenotypes. This urges the need for more rapid and more 
consistent methods.
Hereby, a computer
program that identifies C. elegans individuals in digital images would 
provide an automated solution for the recognition task, thus vastly 
improving the time-cost and accuracy with respect to manual identification 
and converting the images into manageable data.\\

The focus of this work is to design and implement an image processing
methodology to detect C. elegans worms shapes in microscope images.
It is easy for a human to find the worm conformation, rotation and direction. In this project we will only 
try to find the conformation. We will investigate if is possible to detect and fit the conformations of worms 
in an automated way and if this can be achieved faster than manual identification.\\

In this project we consider larval worms in microtiter plates. The larvae are 
grown in liquid culture, which causes the microscope image background to be 
very clear. However, worms overlap frequently. Ordinary bright-field images are 
used for the morphology. The implementation is adjusted to \emph{Endrov},
an open-source plug-in architecture aimed for image analysis and data processing, developed and used at the 
\emph{Department of Bioscience and Nutrition, Karolinska Institute},
where this project is carried out.

\begin{itemize}
\item General Objective
  \begin{itemize}
  \item To design and implement an image processing methodology to detect
    and fit the shape of worms in digital images.    
  \end{itemize}
\end{itemize}
\begin{itemize}
\item Specific Objectives
  \begin{itemize}
  \item To design an algorithm based on image processing techniques that
    receives images of worms in liquid culture as input, and outputs
    fitted shapes of these worms.
    \begin{itemize}
    \item To review the background on image segmentation techniques.    
    \item To design a shape descriptor and a rasterization method to
      represent worms in numerical terms.
    \item To review the background on shape matching and object recognition and
      propose a matching approach.    
    \end{itemize}
  \item To implement the designed algorithm as a plug-in for \emph{Endrov}
  \end{itemize}
\end{itemize}

\section{Earlier Work}

C. elegans has been deployed quickly in genetic screens and chemical
screens, as mentioned in \cite{automated}. In the previously mentioned paper,
the strategies for automated analysis of C. elegans locomotion are
divided in three groups according to their methodological approach:
Tracking overall behavior, detection and measurement of distinct behaviors and
measuring the complete behavioral repertoire using large parameters sets.
All of these strategies are based on an initial processing step that performs the detection of the worm shapes in the images. This usually involves the extraction of the worms from the background (segmentation), reduction to a skeleton and parameterization of worm outlines.

Reduction to skeleton and subsequent parameterization have become a standard method. 
However, since the image properties such as lighting, noise and clutters 
(e.g. worm tracks and eggs), can vary strongly from one image to another and 
the segmentation depends directly on the visual context, the  
parameters for this process are highly variable. The segmentation methods 
that are usually used 
on worm images are thresholding, morphological closing, 
hole filling, and their combinations.
It is stated in \cite{automated} that among those programs that track 
multiple worms, few attempt to resolve the problem that arise when 
worms interact or when individual worms coil up on themselves, which may
severely affect the individual worm identification. In \cite{huang} is indicated
that programs and algorithms are being developed to address this problem.\\

The  numerical description of a worm shape or worm parameterization 
determines the range of possible shapes a worm may adopt. Diverse 
representations have been used from one approach to another. The 
most common is the reproduction of an abstract shape, normalized for position, 
orientation, scale and parameterization of the 
worm skeleton.\\

Very recent studies present new approaches for detecting individual worms
in cluttered clusters. Riklin Raviv et al. \cite{individual1} present an approach for
 extracting cluttered objects based on their morphological properties. This study addresses
the problem of untangling C. elegans clusters in high-throughput screening experiments.
The method is based on concepts from machine learning and graph theory. The worm
skeleton is used as shape descriptor. 
The clustered worm segments are represented
as graph vertices and then a search for more likely worm paths in the graph 
is carried out. The detection of the most likely worm descriptors within the graph
search is guided by a probability distribution, defined by a low dimensional feature 
space that captures the worms' variability. This probabilistic shape model and a 
similar worm detection approach was first
presented by W\"{a}hlby et al. in \cite{individual2}.


There are many studies in computer vision dealing with the automated
analysis of C. elegans and nematodes in general. Most studies are
based on worm locomotion, so the process of identification and 
tracking is performed by the simultaneous analysis of a set of images, rather
than just one. There is a standard general strategy followed for an initial
worm identification step consisting in segmentation, skeleton reduction and
shape parameterization, while the matching strategies depend on the approach
and normally involves image sets and not individual images, as explained.
Although some automated worm detection approaches are able to identify
individual worms and group of worms, few of them attempt to overcome the 
problem of worm interaction and none solves it successfully.

\section{Structure of Thesis}
This document is divided the following way:
\begin{itemize}
\item \textbf{Chapter 1: Introduction}\\
  The motivation and purpose of the thesis is discussed. Then, earlier work
  on worms detection is presented, pointing out different approaches, achievements
  and current problems.
\item \textbf{Chapter 2: Theoretical Framework}\\
  The theory related to the problem and the solution is explained. Each topic is described,
  and the different studied approaches are addressed.
\item \textbf{Chapter 3: Methodology}\\
  The proposed solution is discussed and presented as a methodology. Then, each step 
  involving the methodology is explained thoroughly, comprising a description of the 
  approach and some implementation details.
\item \textbf{Chapter 4: Experiments and Results}\\
  A series of experiments to test the performance of the solution are presented. 
  The purpose and characteristics of the experiments are described. Then, the results are shown and discussed.
\item \textbf{Chapter 5: Conclusions and Future Work}\\
  The conclusions obtained from the discussion of the results are presented. 
Suggestions for future work are given.
\end{itemize}
