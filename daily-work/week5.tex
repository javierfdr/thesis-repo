\documentclass{article}
\usepackage{geometry} % see geometry.pdf on how to lay out the page. There's lots.
\geometry{letterpaper} % or letter or a5paper or ... etc
% \geometry{landscape} % rotated page geometry
\RequirePackage{amsmath}
\usepackage{listings}

\usepackage{graphicx}
\usepackage{amssymb}
\usepackage{epstopdf}
\usepackage{longtable}
\usepackage{hyperref}
\usepackage{color}


\title{Daily Thesis-work Report}
\author{Javier Fern\'andez}
%\date{17 de Febrero de 2009} % delete this line to display the current date

\begin{document}

\maketitle
%\newpage
%\tableofcontents
%\listoffigures
%\listoftables
%\newpage

\section{General topics addressed}

\begin{itemize}
\item Combinatorial optimization in image analysis paper review
\item Splines: fast and accurate implementation
\item Curve fitting / Spline fitting techiques
\end{itemize}

\section{Combinatorial Optimization in image analysis}
Rally nice explanation on topics like \textbf{local search, simulated annealing, genetic algorithms}\\


\subsection*{GA and SA}

Big reference to \emph{An integrated genetic algorithm} which envolves several adds to make it
more powerful than a simple genetic algorithm. Is mentioned the importance in having
a rich pool of solutions in a simple genetic algorithm, they rely on it. The convergence of the
GA improved considerably after the incorporation of a knowledge based mutation operator and 
meta-level genetic operators (took the same time to generate a new generation that one iteration
of SA). As usual local search is 15 times faster, finds acceptable solutions but the minimun
of the cost function is higher so the quality of solutions is lower.\\


\subsection*{Image Segmentation}
Image segmentation is a clustering procedure applied to a matrix representation of the
image. It typically partitions the spatial domain of an image into mutually exclusive
subsets, called regions. Each region is uniform and homogeneous with respect to some
property such as tone or texture, and its property value differs in some significant way
from the property value of each neigbouring region. The segmentation of an image into
meaningful parts is a key step in nearly every image analysis problem. It is crucial to the
successful identification of the depicted targets, and to the accuracy of target analysis such
as shape and area.

\subsection*{Target Recognition}
Model-based target recognition involves the detection and classification of instantiations
of predetermined patterns or target models in an image. This involves the estimation of the
parameters of a model -to- image transformation that results in the observed image

\subsection*{Contour Matching}
Optimization of a \textrm{distance matching} function that measures the distance of the
image shape to a set of predefined shapes or template contour.\\

Algorithms involved SA, GA and Markov Random filed polar model.

\subsubsection*{GA Approach}
Generating a pool of random templates all around the image to avoid local minima.
In this scheme the size and shape of the model contour adapt to the local image evidence.


\subsection{Important concluding remarks}
\begin{itemize}
\item Of all modern optimization techniques, simulated annealing is most frequently applied to
solve large combinatorial problems in image analysis. It produces good quality solutions,
but converges rather slowly

\item Genetic algorithms produce good quality solutions when the population size fully repre-
sents the diversity of the underlying search space. They converge faster when the local
search (mutation) procedure is constrained by a priori knowledge of the image structure
(Bhandarkar et al., 1994).

\item Local search generally yields the fastest rate of convergence, but also produces the lowest
quality solutions because the algorithms easily get trapped in local minima

\item Mean field annealing is considerably more efficient than simulated annealing, and produces
nearly the same quality of the final solution 

\end{itemize}

\subsection{Worm Clustering}
The worms are segmentated in cluster of worms. Each cluster is found starting in the base points (found after thinning)
 and reaching each other of the base point. The final result is a totally connected skeleton that involves 1 or more
worms\\

Since the skeleton obtained using the thinning algorithm does not have its extreme in the actual worm extreme then 
the skeleton is expanded to match more likely base-extreme points.
This is done this way:
\begin{itemize}
\item Step 1: Get a base point.
\item Step 2: Find the previous skeleton point and calculate the maximal
  directional pixel based on the distance transformation array.
\item repeat Step 2 until a border pixel or a background pixel is found.
  If the pixel is border then is the base, otherwise the previous one is 
  set as base point
\end{itemize}


\subsection{Finding contour in 1-worm WormCluster}
The contour was found selecting one contour pixel and following the 
1-distance-pixel contour recursively until no more are found.\\
To find efficiently the initial pixel a base pixel is selected. If this
is contour is the initial pixel, if not then the 8-neighborhood is checked.
If no one of this is contour then the one with fewest distance value is selected
and the directional pixels are analized. This has to ensure to find a contour
pixel given the worms cilindrical and not that wide shape.

\subsubsection{Many worms Cluster}
The actual implementation is located in the \emph{Thickening} class and is 
aimed to rebuild the shape iteratively. This is starting from the 
skeleton points and with the highest distance value and iteratively add
the new neighbors that belong to the shape to the list. The pixels are
accepted as part of the shape if they have the same current distance value
when analized. \\


\subsection{Worm cluster Path  guessing}

Different ideas were tried, until the definitive implementation was obtained. These are:
\begin{itemize}
\item \textbf{Idea: } Follow the first directional neighbor. Modified as: follow the best(distance transformation) directional neighbor.\\
  \textbf{Result: }Taking wrong paths constantly. The best directional just tends to mantain the last taken direction is able to change
  from direction drastically, thus taking wrong decisions on crossroads or intersection squares.

\item \textbf{Idea: } Add to the directional neighbors the 4 diagonal directions in order to make it follow the must common diagonal direction.
  Modified as: chose the pixel which tend to be the best pixel in the next $m$ steps
  \textbf{Result: } The cross-neighbor representation of the skeleton makes too hard to guess if picking some giving cross point will tend 
  to follow the diagonal direction stablished. This is: can not (or is hard) to make it follow the direction

\item \textbf{Idea: }Keep the last n steps following the skeleton. This is built only using cardinal or cross coordinates, thus the directions
  are these 4. Follow the neighbor(among cross neighbors) that goes in the higher direction value.\\
  \textbf{Result: } Very good for obvious wrong turn-arounds like the cross-shape worms cluster. Not accurate for many worms cluster with different
  positions.

\item \textbf{Idea: }Keep the last n steps following the skeleton. The best neighbor heuristic is constructed as following:
$$best(neighbors(p)) = MAX_{nn \in neighbors(p)}(directions[nn]+(dt[nn]*cv))$$ where $cv$ is the compensation value for the distance
transformation. The current value is fixed as 2 for $nn=15$ and has to be generalized. It actually tries to give importance to the distance
transformation that tend to have a difference no higher than 3.\\
The idea behind this is to make the followed to reach the center of the intersection squares or crossroads to take the next decision, in order to
avoid taking surrounding paths. This is necessary because in some cases, the accumulated sum of two given values in the last n steps is too close, which
will point to a diagonal direction growing, and thus not giving enough information to decide direction.
\emph{e.g} when the worm is sort of straight line going in some diagonal direction

\textbf{Result: }Very good result. In the sample basic image every worm was guessed correctly
\end{itemize}


\subsection{Splines}
Great link for many Java applets. Gotta check if the source code for
cardinal spline is usable. It seems to work really well.
\url{http://www.netgraphics.sk/cardinal-spline}\\

Seems to implement cardinal splines: \url{http://prefuse.org/doc/api/prefuse/util/GraphicsLib.html}. it's called \textbf{Prefuse} and information visualization
toolkit. Seems interesting and useful. BSD license by the way.


\end{document}

