\documentclass{article}
\usepackage{geometry} % see geometry.pdf on how to lay out the page. There's lots.
\geometry{letterpaper} % or letter or a5paper or ... etc
% \geometry{landscape} % rotated page geometry
\RequirePackage{amsmath}
\usepackage{listings}

\usepackage{graphicx}
\usepackage{amssymb}
\usepackage{epstopdf}
\usepackage{longtable}
\usepackage{hyperref}
\usepackage{color}


\title{Daily Thesis-work Report}
\author{Javier Fern\'andez}
%\date{17 de Febrero de 2009} % delete this line to display the current date

\begin{document}

\maketitle
%\newpage
%\tableofcontents
%\listoffigures
%\listoftables
%\newpage

\section{General topics addressed}

\begin{itemize}
\item Distance transform approaches (research)
\end{itemize}

\section{On distance transformation}

Metric: the distance between two points is the sum of the (absolute)
differences of their coordinates. The taxicab metric is also known as 
rectilinear distance, L1 distance or $\ell$1 norm (see Lp space), city block
distance, Manhattan distance, or Manhattan length, with corresponding
variations in the name of the geometry. 
\url{http://en.wikipedia.org/wiki/Taxicab_geometry}\\

Different Distance metric: \url{http://www.mathworks.com/access/helpdesk_r13/help/toolbox/images/morph15.html}\\
\url{http://www.mathworks.com/access/helpdesk/help/toolbox/images/bwdist.html}


\textbf{City block: }
The city block distance metric measures the path between the pixels 
based on a 4-connected neighborhood. Pixels whose edges touch are 1 
unit apart; pixels diagonally touching are 2 units apart. \\

\textbf{Chessboard: }The chessboard distance metric measures the path 
between the pixels based on an 8-connected neighborhood. Pixels whose 
edges or corners touch are 1 unit apart.


\section{Fast Distance transform in two scans using 3x3}
Distance transformation (DT) is to convert a digital binary image 
that consists of object (foreground) and non-object (background) pixels
into another image in which each object pixel has a value corresponding
to the minimum distance from the background by a distance function

\textbf{NICE papers to take theoretical documentation}

\subsection{Implementation of Manhattan and Chessboard distance transformation}
Following the procedure described on the paper it was donde a two-scan process
to fill the distance matrix. The first one is done from left to right, top to bottom,
(forward scan). The second one, backward scan, is done from right to left, bottom to top.
The neighbors are calculated depending on the distance metric. The corresponding 
neighborhoods $N_1$ and $N_2$ are used in each scan respectively.\\

The obtained result give an aceptable distance transformation. But looks like it
needs some overwork to find the skeleton appropriately.

\begin{itemize}
\item Interesting disertion about why one or the other is good at:
 \textbf{ Ridge points in euclidean distance maps}
\end{itemize}

\subsection{Euclidean distance transformation: discrete and two scans}

Implemented. The method is analyzable, but yet a bit tricky. It resulted
wide slower than the others (optimizable though). The final result is not that
impresive. Manhattan works better.


\section{Skeletonization}

Nice introduction about why is skeelton used can be found at the introduction
of the paper \textbf{A one pass thinning algorithm and its parallel
implementation}

\subsection{A skeletonization algorithm bby maxima tracking euclidean dt}

nice definition of distance transform on page 332.\\

\textbf{On two scan distance transform: }\\

Another algorithm only needs two operations in opposite scans
of the picture: one scan is in the left-to-right top-to-bottom
direction, and the other is in the right-to-
left bottom-to-top direction. 116"t7) This does not
require iterations and is efficient on conventional com-
puters. Other algorithms apply the morphological
erosions.t 18-20)\\

\textbf{Why Euclidean over Chessboard and City-block: }
The city-block or chessboard distance measures are
sensitive to the rotations of an object, but the Euclidean
distance measure is rotation invariant. However, its
square root operation is costly and the global nature
of distance transformation is difficult to decompose
into small neighborhood operations because of the
nonlinearity of Euclidean distance variations

\textbf{Good algorithm}
Very followable algorithm. Analyze well each stage to understand why is it
taken that way.\\

In the paper: \textbf{Using ssm} in the introduction and other parts they
introduce well why is it the medial axis descriptor good for representing
skeleton.\\

\subsubsection{Apex and base points}
The apex and base points where computed following the algorithm presented in the paper\\
The base points are such who have at least 4 zeros around, this was changed to 5 because
according to the worm shape there were many base points around the worm shape.
This generated well the real base (extreme) points of the worms with some issues: some
real base points are missing (not much) and there are some points in places that do not
correspond to an extreme of the worm. Need some refactoring here.
\\

The apex points are such who are the local maxima in their 3x3 (8-n) neighborhood. 
These are generated well. The small width of the worms generates a 2-pixel width
in some areas. This will be reduce with a pixel-reducer algorithm following the skeleton.

\subsubsection{UpHill}

The uphill generation looks for relative apex\_base points to construct a connected
skeleton. The connection is done following the \emph{best possible path} which is 
assumed to be given by the higher value pixel in the three 45 degrees directional pixels.\\
This connected well many pixels approaching the shape well to a connected skeleton, but there
are still missing pixels. The downhill processing should fix this.\\

Another problem regarding this is the apperance of \emph{strange} or \emph{ilogic} connections
between wrongly generated pixels (normally base) and the partial skeleton.\\

\subsubsection{Downhill and Uphill problems}
Uphill gives the following problems: 
\begin{itemize}
\item When a pixel is followed just by one and only one pixel it follows
it in the second pixel direction. That doesn't allow to explore empty
fields, just keeps follows the skeleton. 

\item When it has no pixel around then is marked, hoping that the downhill
procedure can connect it with a partially built skeleton on UpHill.

\item When is sorrounded by many then follows the best one with a directional
neighbor. That is ok.


\end{itemize}

\subsubsection{Thining (reduction)}
The implemented reduction algorithm does not work properly. Disconnects
some parts with the \emph{elder} implementation of UpDownHill2. This
last was changed, deleting the nonConnected check for the 1pixel intermediate
neighbor. After this the obtained skeleton is thicker but totally connected.\\

\begin{itemize}
\item I am looking for literature to implement a best thining algorithm
  that works properly.
\item When explaining take an eye to page 164 (anchor skeleton). They
  talk about thinning without touching special skeleton points, such 
  as centres of maximal discs. NEEEH
\item The need for thining skeleton comes for several things:
  \begin{itemize}
  \item It makes easier and viable the segmentation of general image 
    in sections. Otherwise it requires many considerations
    and it will be ambigous to define crossroads points
  \item Once segmentated, the "mixed" skeletons will be part of a unique
    worm representation. So it will be easy to calculate the optimization
    function cause it will depend on only one point each time.
  \end{itemize}
\item There is an algorithm in \textbf{Fast parallel algorithm for thining}.
  It seems to require to many iterations and calculations that are not 
  explicitly mentioned, also is kinda dummy. Though, the 
  obtained skeleton seems perfect, no missing points and totally connected.
\end{itemize}

\subsubsection{Refine base points}

After UpDownHill2 with TwoStepConnections gives us a totally connected
skeleton we can refine the previously calculated base points.\\
A base point have to meet every of the following conditons

\begin{itemize}
\item It has to have at least one skeleton neighbor. (due to the total
  connection)
\item If it has more than one neighbor and one of those is a base point//
  \textbf{This is not taked into consideration}

\end{itemize}

\subsubsection{A fast algorithm for thinning digital patterns}

Implementing the algorithm.\\

\subsubsection*{Description}
It consists of two subiterations: one aimed at
deleting the south-east boundary points and the north-west
corner points while the other one is aimed at deleting the
north-west boundary points and the south-east corner
points. End points and pixel connectivity are preserved.
Each pattern is thinned down to a "skeleton" of unitary
thickness.\\

\subsubsection*{Procedure}
The algorithm removes the border or contour pixel that do not belong to 
a skeleton representation following the conditions that will be explained
below. \\
A distance transformation of the image is used to find the current
contour pixel for each run of the algorithm. A distance counter is started
at 1 (border pixel) an augmented in each iteration, so pixels 
analized per iteration are the one who correspond to the distance counter
in the image transformation (the contour pixels).\\

The conditions checked in each iteration are:
\begin{itemize}
\item First iteration
  \begin{itemize}
  \item $2 \leq B(p1) \leq 6$ where $B(p)$ is the number of shape-pixels
    neighbors
  \item $A(p1) = 1$ The number of consecutive $0$-$1$, compared clockwise
    among the circular neighbors.
  \item $p2*p4*p6 = 0$
  \item $p4*p6*p8 = 0$
  \end{itemize}
\item Second iteration
  \item First two above
  \item $p2*p4*p8 = 0$
  \item $p2*p6*p8 = 0$
  \end{itemize}

\subsubsection{Detection of base-points}
To detect the baes points the following conditions were taking into account
\begin{itemize}
\item The pixel has only one neighbor. In a connected 1-width skeleton
  any shape point with only one neighbor is inmediately a base point.
\item The 1-pixel width reduction method does not consider diagonal connections,
  so just a 4-neighbors connections is taken into account. This makes that when
  created a diagonal a manhattan two step is done. So the following condition
  has to be accomplished for having a base point: There is one and just one 
  \emph{group} directional neighbor. The groups are conformed by consecutive
  neighbors taken them clockwise.
\end{itemize}

Works perfectly and in a real-worm shape the points detected as base points are
just the right ones. If there is still some noise, these will be recognized
as base points when corresponding.



\end{document}

