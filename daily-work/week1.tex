\documentclass{article}
\usepackage{geometry} % see geometry.pdf on how to lay out the page. There's lots.
\geometry{letterpaper} % or letter or a5paper or ... etc
% \geometry{landscape} % rotated page geometry
\RequirePackage{amsmath}
\usepackage{listings}

\usepackage{graphicx}
\usepackage{amssymb}
\usepackage{epstopdf}
\usepackage{longtable}
\usepackage{hyperref}

\title{Weekly Thesis-work Report}
\author{Javier Fern\'andez}
%\date{17 de Febrero de 2009} % delete this line to display the current date

\begin{document}

\maketitle
%\newpage
%\tableofcontents
%\listoffigures
%\listoftables
%\newpage

\section{General topics addressed}
\begin{itemize}
\item Adding \emph{rm-tex.sh} script for deleting unnecesary trash files 
  created by \LaTeX
\item Using \emph{Endrov} flow view. Try general filters, emphasis on Thresholding.
\item Rasterization and triangle Rasterization research
\item Plugin implementation (basic)

\end{itemize}


\section{Main per-topic Issues and solutions}
\subsection{Endrov flows}

\subsubsection*{Summary}

\begin{itemize}
\item The filters are working properly. Evaluating on Thresholding gives
      a clean input. \\
      \textbf{Solution: } Was necessary to select the
      environment/(the flow) to make the images findable.
\end{itemize}

\subsubsection*{Remaining problems}
\begin{itemize}
\item Cannot view the result of the filtering, i.e. pass the output
      to a channel or a proper place.\\
      \textbf{Error: } \texttt{Trying to overwrite data that has not been 
        autogenerated} $\Rightarrow$ \textbf{Solved: } Take a channel 
      a rename by just clicking on it and rewriting 
\end{itemize}

\subsection{Rasterization research}

\subsubsection*{General ideas}

\begin{itemize}
\item Rasterization: Process of rendering \emph{vector} information
  to convert it into a raster format. vectorized image $\Rightarrow$ bitmap.\\
  In a nutshell is the process of computing the mapping from scene geometry
  to pixels.
  
\item Rasterization is based in shape by shape analysis and rendering. Ray
  tracing can involve more than one object at the same time and is faster
  to compute ilumination (not quite if GPU are well used).
  
\item \textbf{Bresenham's Algorithm}:
  \begin{itemize}
  \item Supposing $(x_0,y_0)$ and $(x_1,y_1)$ the endpoints of a line. And
     the standard pixel convention: pixel increase in the down and right
     directions, then:
     $$y=\frac{y_1-y_0}{x_1-x_0}(x-x_0) + y_0$$
     And this implies that every ideal $y$ for succesive integer values of
     $x$ can be computed as:
     $$y_n = y0 + m*n \Rightarrow m =\frac{y_1-y_0}{x_1-x_0} $$

  \item There is an error value due to the vertical distance between the rounded
    and the exact $y$ values. The error increases gradually by the slope.
    If the error exceeds $0.5$, the rasterization $y$ is increased by $1$ 
    (next row) and the error decremented by $1.0$.\\
    This is only for right-down direction

  \item \textbf{Generalization}: 
    \begin{itemize}
    \item 
      If lines goes left-down ($x_0>x_1$) instead of right-down ($x_0<x_1$)
      then $swap(x0,x1)$
    \item
      If lines goes up then check $y_0>y_1$ and if true then step $-1$ instead
      of $1$
    \item
      If slope $m$ is greater than $1$ then reflect the steep line across the
      line $y=x$ to obtain a line with a small slope. To do this
      $swap(x_0,y_0)$ $swap(x_1,y_1$) perform operations and $plot(y,x)$
    \end{itemize}

  \end{itemize}

\end{itemize}

\subsubsection*{Papers and links}
\begin{itemize}
\item \textbf{Scan-conversion: }\url{http://www.devmaster.net/articles/software-rendering/part3.php}\\
      Drawing a polygon as a set of horizontal lines.
\item \textbf{Basic T.Rasterization: }\url{http://joshbeam.com/articles/triangle_rasterization/}
  Basic triangle rasterization C++ (non-beautiful code)
\item \textbf{Bresenham's line algorithm} \url{http://en.wikipedia.org/wiki/Bresenham\%27s_line_algorithm}
\end{itemize}


\section{Next-week suggestions}

\begin{itemize}
\item Complete the plugin implementing the triangle rasterization 
  \begin{itemize}
  \item Try using int as triangle points
  \item Design a good class scheme  
  \end{itemize}
\end{itemize}


\section{Triangle Rasterizer implementation}
The implemented algorithm is the \emph{Scan-conversion} previously refered which consists
in drawing the polygon (the triangle) as a group of horizontal lines.
The bressenham algorithm was not necessary then.\\

\emph{Algorithm:}
Find the longest segment of the triangle in any a given axis (let's suppose is the y axis).
Draw horizontal lines between the short segments and the long segment.

\end{document}