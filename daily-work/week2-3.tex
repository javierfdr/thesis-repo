\documentclass{article}
\usepackage{geometry} % see geometry.pdf on how to lay out the page. There's lots.
\geometry{letterpaper} % or letter or a5paper or ... etc
% \geometry{landscape} % rotated page geometry
\RequirePackage{amsmath}
\usepackage{listings}

\usepackage{graphicx}
\usepackage{amssymb}
\usepackage{epstopdf}
\usepackage{longtable}
\usepackage{hyperref}
\usepackage{color}


\title{Daily Thesis-work Report}
\author{Javier Fern\'andez}
%\date{17 de Febrero de 2009} % delete this line to display the current date

\begin{document}

\maketitle
%\newpage
%\tableofcontents
%\listoffigures
%\listoftables
%\newpage

\section{General topics addressed}
\begin{itemize}
\item Researching on worm parameterization: the idea is to find a
  general representation for the worms that can be easily moved,
  scaled and bent. (I believe that's the shape descriptor)
\end{itemize}

\section{Main per-topic Issues}
\subsection{Shape descriptions theory}
Interesting short briefing of shape descriptors. Lists some common simple 
descriptors: Area, perimeter, compactness-circularity, Eccentricity, Elongation,
rectangularity. \textbf{(elongation and rectangularity are interesting)}.\\
Centered more on region-based descriptors than on boundaries-based ones.
\\
\url{http://homepages.inf.ed.ac.uk/rbf/CVonline/LOCAL_COPIES/MORSE/region-props-and-moments.pdf}

\subsection{Curve fitting}
Curve fitting is the process of constructing a curve, or mathematical function, that has the best fit to a series of data points, possibly subject to constraints.\\
It adds some nice suggestions:
\begin{itemize}
  \begin{itemize}
  \item $y = ax^3 + bx^2 + cx + d$ In polynoms to fit curve the degree of 
    the polynom $(n)$ sets the number of\emph{ constraints} being fitted.
    Each constraint can be: a point, an angle, a curvature.
    \item \textbf{Runge's phenomenon}: high order polynomials can be highly oscillatory
  \end{itemize}
\item Spline and B\'ezier Curves. Tend to avoid Runge's phenomen, unlike
  polynomial interpolation.
\item \textbf{Fitting}
  \begin{itemize}
  \item Algebraic analysis: "fitting" usually means trying to find the curve that minimizes the vertical (i.e. y-axis) displacement of a point from the curve (e.g. ordinary least squares).
  \item Graphical and images apps: fitting seeks to provide the best visual fit; which usually means trying to minimize the orthogonal distance to the curve (e.g. total least squares), or to otherwise include both axes of displacement of a point from the curve.  
  \end{itemize}
\end{itemize}

\url{http://en.wikipedia.org/wiki/Curve_fitting}

\subsection{Contour based shape descriptors}
Great link for reading an overview of contour-based shape representation and
description.\\
\url{http://www.engineering.uiowa.edu/~dip/LECTURE/Shape2.html}\\

\begin{itemize}
\item \textbf{B-Spline: }Seems to be fastly executed. With a n=3 creates
  a very accurate triangulated countourn\\

Interactive splines: \urlh{ttp://www.ibiblio.org/e-notes/Splines/Intro.htm}
\textbf{Cardinal Splines: }They pass smoothly by every point selected.

\end{itemize}

\subsection{Triangle mesh in figures, easily bent and deformed}
In hd as: \emph{As-rigid-as-possible.pdf}
Web link: \url{http://portal.acm.org/citation.cfm?id=1073204.1073323}\\

Shape manipulation techniques fall roughly into two categories.
One is to deform the space in which the target shape is embedded;
the other is to \emph{deform the shape while taking its structure into
account}.\\

Our goal
is to introduce internal model rigidity into shape manipulation.
However, instead of using physically based models, we use
simple geometric approach similar to a technique used in [Alexa
et al. 2000]. \\

\textbf{The Process}
\begin{itemize}
\item Silhouette tracing with marching squares algorithm
\item Generates a triangulated mesh inside the boundary. Better manipulation
  results achieved using near-equilateral triangles of similar sizes 
  across the region. They use [Markosian 1999], that starts with a standard
  constrained Delaunay triangulation, iteratively refining the mesh.
\item Resize managing rotation and scale.
\end{itemize}

\subsection{Delaunay Triangulation (mesh creation)}
A Delaunay triangulation for a set $P$ of points in the plane is 
a triangulation   $DT(P)$ such that no point in $P$ is inside the 
circumcircle of any triangle  in $DT(P)$. Delaunay triangulations maximize 
the minimum angle of all the angles of the triangles in the triangulation; 
they tend to avoid skinny triangles.\\

For manipulating delaunay triangles
\url{http://www.cs.cornell.edu/home/chew/Delaunay05.html}

\subsection{State of the art in shape matching}
Describes several state-of-the-art techinques in shape matching.\\
\url{http://books.google.com/books?hl=en&lr=&id=2SPuMz-BG-4C&oi=fnd&pg=PA87&dq=shape+matching&ots=Q1qi7dVkhI&sig=erssczZ6x2A73Uw_h77X4NDYu9U#v=onepage&q=shape%20matching&f=false}\\

Interesting briefing on contourn matching techniques:
\url{http://homepages.inf.ed.ac.uk/rbf/CVonline/LOCAL_COPIES/GDALYAHU/gdalyahu.html}

\subsection{Marching Squares}
Contour detection process. Very simple process, can be hard to determine one 
worm over another.
\url{http://en.wikipedia.org/wiki/Marching_squares}
\url{http://users.polytech.unice.fr/~lingrand/MarchingCubes/algo.html}

\subsection{2D Shape deformations}
Look in HD: \emph{2D Shape deformation.pdf}


\subsection{Free form deformation}
Checkout free form \emph{.pdf} on HD.\\
Also: 
\url{http://www.docstoc.com/docs/7613261/Accelerating-Accurate-B-spline-Free-form-Deformation-of-Polygonal}



\subsection{Strategies for automated analysis of C.Elegans}
Check the Homonymus .pdf stored in HD.\\

All automated vision programmes for worms follow a
similar processing strategy which usually involves extrac-
tion of the worm from the background (segmentation)
followed by reduction of the worm to a skeleton \\


\subsection{A trainable method for parametric shape descriptors}
Not much nice things. Method for finding parametric shape descriptors, tested un C.Elegans.
Does not give anything interesting for the moment. Maybe it will help to completing 
the paremeterisation.

\subsection{Chamfer matching}
Interesting distance transform method. Computationally unexpensive, not hard to implement.
Requires a well defined set of templates. It will be useful then if the variations over the shape 
descriptor are enoughly well performed by the optimization algorithm. 

\subsection{Skeleton matching}
\textbf{About parameterization: } The skeleton can be considered mathematically as a function.
It can also be represented as polyline (a string of connected straight lines) and the angles
between the segments measured. 

\subsection{Graph matching for object recognition and recovery}
The object recognition is realized through shape matching, by matching the skeleton graph of
the input contour from a deformable contour method (DCM).
Skeleton is used as shape descriptor due to its significant features on the desired representational
properties, such as invariance to object geometric transformations (translation, rotation and scaling)
and reversibility to the original shape.\\
To the shape description is also given widht and length of each part, and location
of convex-parts.

\subsubsection*{Skeleton structure notation}
The skeleton representation is based on the location of the \textbf{Centers of maximal
disks (CMD)}. A skeleton graph is generated and the segments are recognized and splitted.
The different nodes are identified as: \emph{ending node}, \emph{bifurcation node} and
\emph{normal nodes}. A segment with two \emph{bifurcation nodes} is as \emph{primary} segment,
otherwise is normal.\\


\subsection{ A computational model for C. elegans locomotory behavior: On Parameterization}
Check at Uppsala


\subsection{Genetic contour matching}
From genetic contour matching paper in HD.\\


Keywords:\textbf{ Object detection; Chamfermetric; Edge transform; Distance transform; Contour model; Contour matching; Genetic algorithms;
Optimization}\\

   Object recognition can be formulated as an optimization problem. The objective function measures for instance the
evidential support for any particular projection of the parameterized object contour model onto the input image. A genetic
algorithm can be used to find a set of parameters which provide an optimal interpretation of the image in terms of the
model. Preliminary test results demonstrate the feasibility of the proposed approach.\\

The contour matching present involves 3 stages:

\begin{itemize}
\item Binary representation of the input image is created
  by thresholding.
\item The binary image is transformed into a grey-level image
  in which all pixels have a value that corresponds to the 
  distance to the nearest edge (object contour) pixel
\item A G.A. minimizes an average of the values of the pixels
  in the distance image that coincide with the projection
  of an instatiation of a parameterized 2d contour model.
\end{itemize}\\

The objective function used in this study
is based on the observation that a good match is one
where every element of the projected model contour
is spatially near an image contour.\\

Interesting things in defining the shape descriptor and 
how to define an error difference to minimize.
Is presented for star-shapped objects, which the worms
contour are not.




\end{document}