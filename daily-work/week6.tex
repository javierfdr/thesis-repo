\documentclass{article}
\usepackage{geometry} % see geometry.pdf on how to lay out the page. There's lots.
\geometry{letterpaper} % or letter or a5paper or ... etc
% \geometry{landscape} % rotated page geometry
\RequirePackage{amsmath}
\usepackage{listings}

\usepackage{graphicx}
\usepackage{amssymb}
\usepackage{epstopdf}
\usepackage{longtable}
\usepackage{hyperref}
\usepackage{color}


\title{Daily Thesis-work Report}
\author{Javier Fern\'andez}
%\date{17 de Febrero de 2009} % delete this line to display the current date

\begin{document}

\maketitle
%\newpage
%\tableofcontents
%\listoffigures
%\listoftables
%\newpage

\section{General topics addressed}
\begin{itemize}
\item Final phase implementation
\end{itemize}

\section{Spline implementation}
Using Curve API taking from \url{http://sourceforge.net/projects/curves/}\\
Given two base points and a path (including these two) a cardinal spline
is built using the CardinalSpline functionality of the library, and 
included as attribute of a EVCardinalSpline object.\\
The cardinal spline is computed and returned specifying the number of points 
to be obtained

\section{Worm Tessellation}

The FlowBasicRasterizer was deleted, adding a tesselation package. This package
contains a TriangleRasterizer class that performs the process of triangle
rasterization and shape rasterization (maybe shape will be changed). This 
hasn't been programmed yet.\\

Looking for Delaunay triangulation implementations. There are not many:
trying this one: \url{http://www.leebyron.com/else/mesh/}. The library
is called \emph{Mesh} and implements Voronoi and Delaunay triangulation.
Delaunay triangulation minimizes the minimum angle of all the angles of the 
triangles in the triangulation, they tend to avoid skinny triangles.
\url{http://en.wikipedia.org/wiki/Delaunay_triangulation}\\

\textbf{Update: }The mesh library didn't work well. Actually I have my doubts
that the Delaunay triangulation would work for what I want. It seems that
Delaunay doesn't work properly with convex polygons (shapes). A worm shape
is definitely concave ( a line between two points is not always inside the
figure).\\
\textbf{Possible solution: }Look for a polygon triangulation library/algorithm\\

This C code could be adapted: \url{http://www.flipcode.com/archives/Efficient_Polygon_Triangulation.shtml}\\
Link for different implementation ideas \url{http://www.vterrain.org/Implementation/Libs/triangulate.html}\\

\textbf{Possible library: }Poly2Tri seems to be available in Java: Library
to triangulate polygons. \url{http://code.google.com/p/poly2tri/source/checkout}
\textbf{Issues: } Couldn't get it to work properly..Didn't want to compile

\textbf{Attempt: }Reading to implement a tesselation algorithm: Put
an eye on \emph{Siedel} algorithm

\textbf{Trying Ear-snipping method from Endrov.newer.FlipCodeTessellate}:
about Ear Clipping triangulation \url{https://docs.google.com/viewer?url=http://www.geometrictools.com/Documentation/TriangulationByEarClipping.pdf}

\subsection{Final Solution}
Given clockwise generated points of the worm polygon a triangle tesselation
is created using the Ear-snipping method already implemented on
\emph{Endrov}, and the triangles are individually rasterized to obtain a list
of the points contained in the area of the worm.\\

The implementation of triangle rasterization had change due to some fundamental
implementation errors on border cases. Check \emph{week1.tex/pdf}


\end{document}

